\documentclass[a4paper,11pt,oneside]{memoir}

% Castellano
\usepackage[spanish,es-tabla]{babel}
\selectlanguage{spanish}
\usepackage[utf8]{inputenc}
\usepackage[T1]{fontenc}
\usepackage{lmodern} % Scalable font
\usepackage{microtype}
\usepackage{placeins}

\usepackage{float}
\usepackage{footnote}
\usepackage{xr}
\usepackage{array}

\RequirePackage{booktabs}
\RequirePackage[table]{xcolor}
\RequirePackage{xtab}
\RequirePackage{multirow}

% Links
\usepackage[colorlinks]{hyperref}
\hypersetup{
	allcolors = {red}
}

% Bibliography management
\usepackage[numbers,sort]{natbib}

% Ecuaciones
\usepackage{siunitx}
\DeclareSIUnit\gauss{G}
\usepackage{amsmath}
\usepackage{amsfonts}
\usepackage{amssymb}

% Landscape 
\usepackage{pdflscape}
\usepackage{afterpage}

\counterwithout{equation}{section}

% Rutas de fichero / paquete
\newcommand{\ruta}[1]{{\sffamily #1}}

% Párrafos
\nonzeroparskip

% Code style
\def\code#1{\colorbox{black!5}{\texttt{#1}}}

% Imagenes
\usepackage{graphicx}
\newcommand{\imagen}[2]{
	\begin{figure}[!h]
		\centering
		\includegraphics[width=0.9\textwidth]{#1}
		\caption{#2}\label{fig:#1}
	\end{figure}
	\FloatBarrier
}

\newcommand{\imagenflotante}[2]{
	\begin{figure}%[!h]
		\centering
		\includegraphics[width=0.9\textwidth]{#1}
		\caption{#2}\label{fig:#1}
	\end{figure}
}



% El comando \figura nos permite insertar figuras comodamente, y utilizando
% siempre el mismo formato. Los parametros son:
% 1 -> Porcentaje del ancho de página que ocupará la figura (de 0 a 1)
% 2 --> Fichero de la imagen
% 3 --> Texto a pie de imagen
% 4 --> Etiqueta (label) para referencias
% 5 --> Opciones que queramos pasarle al \includegraphics
% 6 --> Opciones de posicionamiento a pasarle a \begin{figure}
\newcommand{\figuraConPosicion}[6]{%
  \setlength{\anchoFloat}{#1\textwidth}%
  \addtolength{\anchoFloat}{-4\fboxsep}%
  \setlength{\anchoFigura}{\anchoFloat}%
  \begin{figure}[#6]
    \begin{center}%
      \Ovalbox{%
        \begin{minipage}{\anchoFloat}%
          \begin{center}%
            \includegraphics[width=\anchoFigura,#5]{#2}%
            \caption{#3}%
            \label{#4}%
          \end{center}%
        \end{minipage}
      }%
    \end{center}%
  \end{figure}%
}

%
% Comando para incluir imágenes en formato apaisado (sin marco).
\newcommand{\figuraApaisadaSinMarco}[5]{%
  \begin{figure}%
    \begin{center}%
    \includegraphics[angle=90,height=#1\textheight,#5]{#2}%
    \caption{#3}%
    \label{#4}%
    \end{center}%
  \end{figure}%
}
% Para las tablas
\newcommand{\otoprule}{\midrule [\heavyrulewidth]}
%
% Nuevo comando para tablas pequeñas (menos de una página).
\newcommand{\tablaSmall}[5]{%
 \begin{table}[H]
  \begin{center}
   \rowcolors {2}{gray!35}{}
   \begin{tabular}{#2}
    \toprule
    #4
    \otoprule
    #5
    \bottomrule
   \end{tabular}
   \caption{#1}
   \label{tabla:#3}
  \end{center}
 \end{table}
}

%
% Nuevo comando para tablas pequeñas (menos de una página).
\newcommand{\tablaSmallSinColores}[5]{%
 \begin{table}[H]
  \begin{center}
   \begin{tabular}{#2}
    \toprule
    #4
    \otoprule
    #5
    \bottomrule
   \end{tabular}
   \caption{#1}
   \label{tabla:#3}
  \end{center}
 \end{table}
}

\newcommand{\tablaApaisadaSmall}[5]{%
\begin{landscape}
  \begin{table}
   \begin{center}
    \rowcolors {2}{gray!35}{}
    \begin{tabular}{#2}
     \toprule
     #4
     \otoprule
     #5
     \bottomrule
    \end{tabular}
    \caption{#1}
    \label{tabla:#3}
   \end{center}
  \end{table}
\end{landscape}
}

%
% Nuevo comando para tablas grandes con cabecera y filas alternas coloreadas en gris.
\newcommand{\tabla}[6]{%
  \begin{center}
    \tablefirsthead{
      \toprule
      #5
      \otoprule
    }
    \tablehead{
      \multicolumn{#3}{l}{\small\sl continúa desde la página anterior}\\
      \toprule
      #5
      \otoprule
    }
    \tabletail{
      \hline
      \multicolumn{#3}{r}{\small\sl continúa en la página siguiente}\\
    }
    \tablelasttail{
      \hline
    }
    \bottomcaption{#1}
    \rowcolors {2}{gray!35}{}
    \begin{xtabular}{#2}
      #6
      \bottomrule
    \end{xtabular}
    \label{tabla:#4}
  \end{center}
}

%
% Nuevo comando para tablas grandes con cabecera.
\newcommand{\tablaSinColores}[6]{%
  \begin{center}
    \tablefirsthead{
      \toprule
      #5
      \otoprule
    }
    \tablehead{
      \multicolumn{#3}{l}{\small\sl continúa desde la página anterior}\\
      \toprule
      #5
      \otoprule
    }
    \tabletail{
      \hline
      \multicolumn{#3}{r}{\small\sl continúa en la página siguiente}\\
    }
    \tablelasttail{
      \hline
    }
    \bottomcaption{#1}
    \begin{xtabular}{#2}
      #6
      \bottomrule
    \end{xtabular}
    \label{tabla:#4}
  \end{center}
}

%
% Nuevo comando para tablas grandes sin cabecera.
\newcommand{\tablaSinCabecera}[5]{%
  \begin{center}
    \tablefirsthead{
      \toprule
    }
    \tablehead{
      \multicolumn{#3}{l}{\small\sl continúa desde la página anterior}\\
      \hline
    }
    \tabletail{
      \hline
      \multicolumn{#3}{r}{\small\sl continúa en la página siguiente}\\
    }
    \tablelasttail{
      \hline
    }
    \bottomcaption{#1}
  \begin{xtabular}{#2}
    #5
   \bottomrule
  \end{xtabular}
  \label{tabla:#4}
  \end{center}
}



\definecolor{cgoLight}{HTML}{EEEEEE}
\definecolor{cgoExtralight}{HTML}{FFFFFF}

%
% Nuevo comando para tablas grandes sin cabecera.
\newcommand{\tablaSinCabeceraConBandas}[5]{%
  \begin{center}
    \tablefirsthead{
      \toprule
    }
    \tablehead{
      \multicolumn{#3}{l}{\small\sl continúa desde la página anterior}\\
      \hline
    }
    \tabletail{
      \hline
      \multicolumn{#3}{r}{\small\sl continúa en la página siguiente}\\
    }
    \tablelasttail{
      \hline
    }
    \bottomcaption{#1}
    \rowcolors[]{1}{cgoExtralight}{cgoLight}

  \begin{xtabular}{#2}
    #5
   \bottomrule
  \end{xtabular}
  \label{tabla:#4}
  \end{center}
}

\usepackage{breakurl}
% Links en footnote 
\let\oldhref\href
\renewcommand{\href}[2]{\oldhref{#1}{#2}\footnote{\burl{\detokenize{#1}}}}














\graphicspath{ {../img/} }

% Capítulos
\chapterstyle{bianchi}
\newcommand{\capitulo}[2]{
	\setcounter{chapter}{#1}
	\setcounter{section}{0}
	\chapter*{#2}
	\addcontentsline{toc}{chapter}{#2}
	\markboth{#2}{#2}
}

% Apéndices
\renewcommand{\appendixname}{Apéndice}
\renewcommand*\cftappendixname{\appendixname}

\newcommand{\apendice}[1]{
	%\renewcommand{\thechapter}{A}
	\chapter{#1}
}

\renewcommand*\cftappendixname{\appendixname\ }

% Formato de portada
\makeatletter
\usepackage{xcolor}
\newcommand{\tutor}[1]{\def\@tutor{#1}}
\newcommand{\cotutorOne}[1]{\def\@cotutorOne{#1}}
\newcommand{\cotutorTwo}[1]{\def\@cotutorTwo{#1}}
\newcommand{\course}[1]{\def\@course{#1}}
\definecolor{cpardoBox}{HTML}{E6E6FF}
\def\maketitle{
  \null
  \thispagestyle{empty}
  % Cabecera ----------------
\noindent\includegraphics[width=\textwidth]{cabecera}\vspace{1cm}%
  \vfill
  % Título proyecto y escudo informática ----------------
  \colorbox{cpardoBox}{%
    \begin{minipage}{.9\textwidth}
      \vspace{.5cm}\large
      \begin{center}
      \textbf{TFG del Grado en Ingeniería Informática}\vspace{.6cm}\\
      \textbf{\@title{}}
      \end{center}
      \vspace{.2cm}
    \end{minipage}

  }%
  \hfill\begin{minipage}{.20\textwidth}
    \includegraphics[width=\textwidth]{escudoInfor}
  \end{minipage}
  \vfill
  % Datos de alumno, curso y tutores ------------------
  \begin{center}%
  {%
    \noindent\large
    Presentado por \@author{}\\ 
    en Universidad de Burgos --- \@date{}\vspace{.6cm}\\
    \noindent \textbf{Tutores}\\[.5em]
    \@tutor{}\\
    \@cotutorOne{}\\
    \@cotutorTwo{}
  }%
  \end{center}%
  \null
  \cleardoublepage
  }
\makeatother

\newcommand{\nombre}{Mario Bartolomé Manovel} %%% cambio de comando

% Datos de portada
\title{Sistema de Navegación Semiautónomo en Interiores}
\author{\nombre}
\tutor{Dr. Alejandro Merino Gómez}
\cotutorOne{Dr. César Ignacio García Osorio}
\cotutorTwo{Dr. José Francisco Díez Pastor}
\date{\today}

\begin{document}

\maketitle
\def\titulo{Sistema de Navegación Semiautónomo en Interiores}
\def\tutor{Dr. Alejandro Merino Gómez}
\def\cotutorOne{Dr. César Ignacio García Osorio}
\def\cotutorTwo{Dr. José Francisco Díez Pastor}
\newpage\null\thispagestyle{empty}\newpage


%%%%%%%%%%%%%%%%%%%%%%%%%%%%%%%%%%%%%%%%%%%%%%%%%%%%%%%%%%%%%%%%%%%%%%%%%%%%%%%%%%%%%%%%
\thispagestyle{empty}


\noindent\includegraphics[width=\textwidth]{cabecera}\vspace{1cm}

\noindent D. \tutor, profesor del Departamento de Ingeniería Electromecánica, Área de Ingeniería de Sistemas y Automática. D. \cotutorOne{} y D. \cotutorTwo{}, profesores del Departamento de Ingeniería Civil, Área de Lenguajes y Sistemas Informáticos.

\noindent \textbf{Exponen}:

\noindent Que el alumno D. \nombre, con DNI 71298657Z, ha realizado el Trabajo final de Grado en Ingeniería Informática titulado <<\titulo>>. 

\noindent Y que dicho trabajo ha sido realizado por el alumno bajo la dirección de los que suscriben, en virtud de lo cual se autoriza su presentación y defensa. 

\begin{center} %\large
En Burgos, {\large \today}
\end{center}

\vfill\vfill\vfill


\begin{minipage}{0.3\textwidth}
\raggedright
Vº. Bº. del Tutor:\\[2cm]
D. \tutor
\end{minipage}
\hfill
\begin{minipage}{0.3\textwidth}
\centering
Vº. Bº. del Tutor:\\[2cm]
D. \cotutorOne
\end{minipage}
\hfill
\begin{minipage}{0.3\textwidth}
\raggedleft
Vº. Bº. del Tutor:\\[2cm]
D. \cotutorTwo
\end{minipage}
\hfill
\vfill

% Author and supervisor
%\begin{minipage}{0.45\textwidth}
%\begin{flushleft} %\large
%Vº. Bº. del Tutor:\\[2cm]
%D. nombre tutor
%\end{flushleft}
%\end{minipage}
%\hfill
%\begin{minipage}{0.45\textwidth}
%\begin{flushleft} %\large
%Vº. Bº. del co-tutor:\\[2cm]
%D. nombre co-tutor
%\end{flushleft}
%\end{minipage}
%\hfill
%
%\vfill

% para casos con solo un tutor comentar lo anterior
% y descomentar lo siguiente
%Vº. Bº. del Tutor:\\[2cm]
%D. nombre tutor


\newpage\null\thispagestyle{empty}\newpage




\frontmatter

% Abstract en castellano
\renewcommand*\abstractname{Resumen}
\begin{abstract}
El objetivo del proyecto es diseñar un sistema de navegación semi-autónomo en espacios cerrados, destinado a la asistencia en vigilancia de seguridad mediante \emph{drones}. Dada una estancia, el \emph{drone} deberá ser capaz de realizar un recorrido por el interior, grabando vídeo que será emitido a un servidor, y transmitiendo su posición dentro del mapa a un responsable de seguridad. El \emph{drone} estará equipado con una RaspberryPi que hará las veces de sistema de control y de transmisor de vídeo. 
Para llevar a cabo la evasión de obstáculos se han implementado dos soluciones, una basada en campos potenciales, y otra, conocida como \emph{Vector Field Histogram}, que complementa a esta primera, solucionando algunos de los problemas que esta presenta. 

La localización en interiores no es fiable mediante GPS, por ello se hará uso de un sistema probabilístico basado en simulación de instancias. Dicho sistema, conocido como filtro de partículas, está basado en el método de Montecarlo como estimador de una variable en base a observaciones del entorno. 

Para lograr la comunicación entre el \emph{drone} y la RaspberryPi, se ha creado una implementación del protocolo \emph{MultiWii Serial Protocol}, el cual permite tanto el envío de comandos, como la solicitud de información a la controladora de vuelo.
\end{abstract}

\renewcommand*\abstractname{Descriptores}
\begin{abstract}
\emph{drone}, vehículo aéreo no tripulado, semi-autónomo, semi-automático, vigilancia, navegación, interior, vídeo, filtro de partículas, campos potenciales, Histograma de Campos Vectoriales, VFH, búsqueda de ruta, MSP, RasberryPi
\end{abstract}

\clearpage

% Abstract en inglés
\renewcommand*\abstractname{Abstract}
\begin{abstract}
The aim of this project is to design a semi-autonomous navigation system in enclosed spaces, destined to aid security vigilance using drones. Given an enclosed space, the \emph{drone} should be able to make its path through it, recording video which will be streamed to a server, and updating its position inside of the enclosed space to the security guard in charge. The drone will be guided making use of a RaspberryPi, which will be controlling the drone and streaming the video. 
To achieve the obstacle avoidance, two different solutions have been implemented, the first based on potential fields, and another one, known as \emph{Vector Field Histogram}, to complement the first one, solving some of its caveats.

Enclosed space location through GPS is not reliable enough, therefore a probabilistic method, based on instance simulation will be used. Also known as \emph{Particle Filter}, this method is based on Montecarlo simulations as an estimator of an unknown variable taking into account multiple observations of the environment.

To achieve communication between the drone and the RaspberryPi, an implementation of the MultiWii Serial Protocol was developed, which allows to send commands and retrieve information from the flight controller.
\end{abstract}

\renewcommand*\abstractname{Keywords}
\begin{abstract}
drone, UAV, semi-autonomous, semi-automatic, vigilance, navigation, indoor, video, particle filter, potential fields, Vector Field Histogram, VFH, path searching, MSP, RaspberryPi
\end{abstract}

\clearpage

% Indices
\hypersetup{colorlinks=true, linkcolor=[rgb]{0.0, 0.3266, 0.7061}}
\setcounter{tocdepth}{2}
\tableofcontents

\clearpage

\listoffigures

\clearpage

\listoftables
\clearpage

\mainmatter
\capitulo{1}{Introducción}

\noindent En este documento se encuentra toda la información relacionada con el Trabajo de Fin de Grado titulado \textit{Sistema de Navegación Semiautónomo en Interiores}.

\noindent En él se puede encontrar la siguiente información:
\begin{itemize}
\item \textbf{Conceptos teóricos}: Ofrecen una base teórica de la que partir, para llevar a cabo el desarrollo completo del proyecto.
\item \textbf{Técnicas y herramientas}: Se trata de las implementaciones de, ó uso dado a, los distintos conceptos teóricos anteriormente descritos.
\item \textbf{Aspectos relevantes del desarrollo del proyecto}: Proporciona información detallada que se ha tenido en cuenta durante las diferentes fases de desarrollo del proyecto.
\item \textbf{Trabajos relacionados}: Se trata de una lista, junto con una breve descripción, de los diferentes proyectos, papers o trabajos relacionados con el proyecto llevado a cabo.
\item \textbf{Conclusiones y líneas de trabajo futuras}: Detalla una serie de posibles mejoras, modificaciones e incluso derivaciones, que pueden surgir del proyecto realizado.
\end{itemize}

\capitulo{2}{Objetivos del proyecto}

Este apartado explica de forma precisa y concisa cuales son los objetivos que se persiguen con la realización del proyecto. Se puede distinguir entre los objetivos marcados por los requisitos del software a construir y los objetivos de carácter técnico que plantea a la hora de llevar a la práctica el proyecto.

\section{Objetivos marcados por Requisitos Funcionales}

\begin{itemize}
\item Diseñar un drone capaz de recorrer un espacio, en el que existan obstáculos, de forma segura.
\item Diseñar un sistema de acceso al drone de forma segura. Tanto para controlarlo de forma remota, como para activa los mecanismos de control automatizados.
\item Diseñar una interfaz web que permita la visualización en tiempo real de la cámara del drone, así como su control remoto por un operador.
\end{itemize}


\section{Objetivos técnicos derivados}

\begin{itemize}
\item Implementar un protocolo de comunicación entre el sistema de control de vuelo de un drone, y una RaspberryPi o similar.
\item Implementar la adquisición de información de una serie de sensores de ultrasonidos mediante una RaspberryPi.
\item Implementar un algoritmo de evasión de obstáculos haciendo uso de sensores de ultrasonidos y del magnetómetro disponible en la controladora de vuelo, que permita el movimiento del drone por el interior de un entorno cerrado. 
\item Implementar un algoritmo de localización \emph{sin} hacer uso de GPS.
\item Implementar una solución de comunicación en tiempo real para vídeo, a través de una aplicación web, haciendo uso de WebRTC.
\item Implementar una solución de control remoto en tiempo real, a través de una aplicación web, haciendo uso de WebSockets.
\item Lograr que dicha solución de control remoto permita la utilización de una emisora, u otro tipo de Joystick con el número de ejes necesario, para controlar el drone de forma adecuada.
\item Hacer uso de protocolos seguros, como SSL/TLS, para el desarrollo de las comunicaciones.  
\item Aplicar metodologías ágiles, como SCRUM, para el desarrollo del proyecto.
\item Utilizar ZenHub como implementación de Kanban, y de Sprints mediante \textit{épicas}.
\item Hacer de Git como sistema de control de versiones.
\item Hacer uso de GitHub como repositorio remoto del sistema de control de versiones.
\item Hacer uso de GitHub como implementación de SCRUM mediante su sistema de gestión de tareas.
\end{itemize}


\section{Objetivos personales}

\begin{itemize}
\item Crear un \textbf{nuevo} sistema de navegación semiautónoma haciendo uso de drones.
\item Realizar una implementación de algoritmos utilizados por empresas líder en el sector de sistemas autónomos.
\item Realizar una aproximación a herramientas utilizadas en un entorno laboral.
\item Profundizar en Python y algunas de sus librerías más utilizadas.
\item Disfrutar en la realización de algo tan extenso como un TFG.
\item Convertir algo que comenzó como un hobby, en una opción de futuro.
\end{itemize}
\capitulo{3}{Conceptos teóricos}


Esta sección auna los diferentes conocimientos teóricos necesarios para la realización del proyecto. A continuación, y en este orden, se explicarán los algoritmos utilizados, protocolos de comunicación y sistemas físicos empleados.

\section{Algoritmia}


\subsection{Filtros de Partículas}

Los Filtros de Partículas son modelos utilizados para tratar de estimar el estado de un sistema que cambia con el tiempo. 
\\Fue definido como \textit{bootstrap filter} en 1993 por N. Gordon, D. Salmond y A. Smith.  Se trata de un método que pretende implementar filtros bayesianos recursivos, haciendo uso del método de Montecarlo, es decir, realiza repetidas medidas del estado para estimar la posición del sistema.
\\ 

\subsubsection{Subsubsecciones}

Y subsecciones. 

\subsection{Campos Potenciales}

A la hora de lograr una navegación segura en un entorno determinado, es necesario implementar un sistema de evasión de obstáculos. 
\\El algoritmo de Campos Potenciales aporta una manera de evitar colisionar con los diferentes obstáculos existentes.

\section{Dispositivos Físicos}

\subsection{Raspberry Pi}


\begin{table}
	\begin{center}
		\rowcolors {2}{gray!35}{}
		\begin{tabular}{l | c}\hline
			\toprule
			Componente & Raspberry Pi 3B\\
			\otoprule
			CPU & BCM2837\\
			Núcleos & 4\\
			Velocidad & 1.2GHz\\
			RAM & 1GB\\
			Coms & Ethernet, WiFi, Bluetooth\\
			USB & 4 (2.0)\\
			GPIO & 40\\
			Consumo máximo & 6.7W\\
			\bottomrule
		\end{tabular}
		\caption{Componentes de una Raspberry Pi 3 Model B}
		\label{tb:raspi3hardware}
	\end{center}
\end{table}

\noindent Una Raspberry Pi es un pequeño ordenador desarrollado en UK por la Raspberry Pi Foundation, con la intención de promover el aprendizaje de informática básica en colegios y países en desarrollo. El modelo base se compone de una única placa de medidas 85mm x 56mm (LxA), y unos 42g de peso.\\Concretamente el modelo empleado es una Raspberry Pi 3B, y consta de los componentes detallados en la tabla \ref{tb:raspi3hardware}



\noindent Su reducido tamaño y bajo consumo lo hacen ideal para este tipo de proyecto. \\En este caso se utiliza bajo una distribución GNU/Linux llamada Raspbian\footnote{Descargable desde: https://www.raspberrypi.org/downloads/raspbian}, basada en Debian. Para tratar de mejorar su rendimiento y reducir al mínimo el consumo, se ha escogido la versión Lite del sistema, es decir, un sistema mínimo sin entorno de escritorio y con la mayor parte de servicios desactivados por defecto.

\begin{figure}
	\centering
	\includegraphics[width=0.9\textwidth]{raspi}
	\caption{Raspberry Pi 3.}\label{fig:raspi3b}
\end{figure}

Una vez descargado el sistema, este ha sido instalado en una micro SD mediante el siguiente procedimiento por consola\footnote{Realizado en OSX, aunque en Linux es muy similar}:
\begin{itemize}
\item\code{diskutil list} Permite localizar el dispositivo en el que se encuentra la tarjeta. En nuestro caso \code{/dev/disk4}
\item\code{diskutil umountDisk /dev/disk4} Permite desmontar el volumen. 
\item\code{sudo dd if=raspbian-stretch.img of=/dev/rdisk4 bs=1m} El comando \code{dd} copia la entrada estándar a la salida estándar. Mediante \code{if/of} se establece el fichero de entrada/salida. Mediante \code{bs} se establece el tamaño de bloque a copiar. Se está utilizando \code{/dev/\underline{\textbf{r}}disk4} en lugar de \code{/dev/disk4} debido a la capacidad de OSX de trabajar con dispositivos en bruto, \textit{raw}, de forma que es posible acceder al dispositivo de forma directa\footnote{Véase \code{man hdiutil}, sección \textit{DEVICE SPECIAL FILES}}, sin almacenar en un buffer la lectura del archivo, proporcionando velocidades de escritura/lectura hasta 20 veces más rápidas.
\end{itemize}


Una vez realizados estos pasos, se puede insertar la microSD en la Raspberry Pi. Para encenderla basta con utilizar el puerto micro-usb de que dispone.\\La Raspberry Pi 3 requiere de una fuente de alimentación capaz de proporcionar 2,5A\footnote{Véase https://www.raspberrypi.org/help/faqs/\#power} para funcionar al máximo nivel de estrés para el procesador y alimentar dispositivos USB.

Sin embargo, este no es estrictamente nuestro caso, véase el apartado \hyperref[subsec:Modificaciones]{Modificaciones}. Se requiere un dispositivo cuyo consumo sea lo más reducido posible, pero que sea rápido en la ejecución, y que muestre poca latencia en operaciones de \code{IO}, que es donde se encuentra el cuello de botella.

Una vez encendida, se accede a ella con el usuario por defecto \code{pi} y la contraseña por defecto \code{raspberry}. Obviamente ambas \textbf{han sido cambiadas} por motivos de seguridad.



\subsubsection{Modificaciones}
\label{subsec:Modificaciones}

\begin{figure}
	\centering
	\includegraphics[width=0.9\textwidth]{sdbench}
	\caption{Benchmark de lector microSD OC.}\label{fig:sdbenchmark}
\end{figure}

\begin{itemize}
\item Se ha desactivado el puerto HDMI para reducir el consumo en \textasciitilde{}30mA: \\Para ello se ha incluido en \code{/etc/rc.local} la línea \code{/usr/bin/tvservice -o}.
\item Se ha overclockeado el lector de microSD a 100MHz, en lugar de los 50MHz por defecto: \\Para ello se ha incluido en \code{/boot/config.txt} la línea \code{dtparam=sd\_overclock=100}.\\Y que arroja los resultados mostrados en la imagen \ref{fig:sdbenchmark}
\item Se han incluido una serie de disipadores para evitar sobrecalentamiento de la placa. Así como un pequeño ventilador de bajo consumo.
\end{itemize} 


\subsection{Controladora de Vuelo}

Una controladora de vuelo (\textit{FC} de aquí en adelante) es un pequeño circuito integrado, que contiene un procesador, una serie de sensores, y una serie de entradas y salidas. 
\begin{figure}
\centering
\includegraphics[width=0.5\textwidth]{flip32}
\caption{Controladora de Vuelo Flip32.}\label{fig:fc}
\end{figure}
La FC se encarga de mantener el sistema de estabilización del drone, tomando medidas de los sensores de que dispone, tales como un acelerómetro, giroscopio, magnetómetro, barómetro... etc. 

En el caso de la FC usada para este proyecto, llamada Flip32 y mostrada en la imagen \ref{fig:fc}, se dispone de un IC MPU-6050, en rojo, con acelerómetro y giroscopio, así como de un barómetro M55611, en verde, y un magnetómetro HMC5883L, en amarillo.
El núcleo de esta pequeña placa es un procesador STM32F103, en cyan, a 72MHz y basado en arquitectura ARM el cual dispone de dos puertos serie, que permiten establecer comunicación entre la FC y otros dispositivos, como emisoras u otros sensores.

El puerto micro-USB de que dispone es utilizado para establecer comunicación entre el configurador de opciones del drone, o en el caso de nuestro proyecto, para hacer uso del \hyperref[subsec:MSP]{MultiWii Serial Protocol}. 


\subsubsection{Entradas}
En el lado derecho de la imagen \ref{fig:fc} pueden verse una serie de pines que actúan como 8 canales de entrada desde el receptor de radio.
Dichos canales de entrada, por defecto, reciben una señal modulada en ancho de pulso o \hyperref[subsec:PWM]{PWM}

 

\subsubsection{Salidas}
En el lado izquierdo de la imagen \ref{fig:fc}, pueden verse otros pines que actúan como salida de señal hacia los controladores de velocidad de los motores del drone (ESC o \textit{Electronic Speed Controller}).
Estos pines de salida, emiten una señal que será interpretada por los ESC del drone, para determinar la frecuencia dada al voltaje que alimenta los motores. En el caso de nuestro proyecto, los ESC disponibles reciben una señal PWM, al igual que la FC desde un receptor de radio.

\newpage
\subsubsection{Sensores}
La controladora de vuelo Flip32 en su versión más completa, dispone de los siguientes sensores:
\begin{itemize}
\item IMU\footnote{Unidad de Medición Inercial.} MPU-6050: Se trata de un circuito integrado compuesto de un acelerómetro de tres (3) ejes y un giroscopio de tres (3) ejes. El acelerómetro mide las fuerzas en los tres diferentes ejes (en \textit{g}) , el giroscopio se encarga de medir la velocidad angular en cada uno de los tres ejes (en degº/s )
\item Magnetómetro HMC5883L: Se trata de un pequeño magnetómetro digital capaz de medir el campo magnético terrestre (en Gauss). Se debe tener en cuenta que según la posición en el planeta, el campo magnético varía entre \si{\gauss{0.25}} - \si{\gauss{0.65}}, así como la declinación magnética de la zona en la que se realiza la medición.\footnote{Disponible en: http://magnetic-declination.com/}
\item Barómetro M55611: Se trata de un altímetro de alta precisión, con resoluciones de hasta 10cm, que funciona midiendo la presión atmosférica (en milibar). Hay que tener en cuenta que los cambios de presión, como los generados por las hélices del drone, hace variar la medida del sensor. Por ello, en el caso de usarlo, se cubre con un pequeño filtro de un material absorbente.
\end{itemize}
\newpage
\section{Protocolos}

\subsection{Pulse Width Modulation}
\label{subsec:PWM}

\externaldocument[5-]{./tex/5_Aspectos_relevantes_del_desarrollo_del_proyecto}

La modulación por ancho de pulso, se basa en medir el transcurso de tiempo entre el flanco de subida de una señal, y el flanco de bajada, tal y como puede verse en la imágen \ref{fig:PWM}.
En este contexto, un receptor de radio recibe una señal de la emisora,  y genera una señal PWM acorde que será transmitida a la FC.
Estas son capaces de entender señales de entre 1000 y 2000\si{\us}. Cualquier valor por debajo, o por encima, haría entrar la controladora en FailSafe\footnote{Al recibir una señal inválida por parte del receptor, la controladora de vuelo puede ser configurada para desactivar el drone, mantener la última medida buena conocida, intentar aterrizar... etc. Este modo es conocido como FailSafe}
En el caso de este proyecto, se ha determinado que no se hará uso de este protocolo para la comunicación entre la Raspberry Pi y la controladora de vuelo, véase \ref{5-sec:MSP_implementation}, y por lo tanto el uso de estos pines queda descartado.

Sin embargo, la comunicación entre la FC y los ESC se realiza mediante este tipo de señal, por ello se ha considerado relevante explicar su funcionamiento.
\begin{figure}
	\centering
	\includegraphics[width=0.9\textwidth]{PWM}
	\caption{Modulación en Ancho de Pulso. Arriba 100\% del ciclo usado. Centro 25\% del ciclo usado. Abajo 50\% del ciclo usado}\label{fig:PWM}
\end{figure}


\subsection{MultiWii Serial Protocol}
\label{subsec:MSP}
\noindent Conocido como \textit{MSP}, se trata de un protocolo para la transferencia de información desde, o hacia una controladora de vuelo.\\\\Generalmente es utilizado durante la configuración del sistema, para transmitir y recibir información de ella.
\\En el caso de este proyecto, se hará uso de este protocolo dado que existe la posibilidad de establecer la recepción de los canales de radio a través de un puerto serie. Es decir, en lugar de utilizar un receptor de radio, se utilizará un puerto serie para obtener la telemetría\footnote{La telemetría es el sistema de medición de magnitudes a distancia; i.e, transmite la información de los diferentes sensores de la controladora de vuelo. }, y establecer las entradas de los canales de radio.


\section{Referencias}

Las referencias se incluyen en el texto usando cite \cite{wiki:latex}. Para citar webs, artículos o libros \cite{koza92}.


\section{Imágenes}

Se pueden incluir imágenes con los comandos standard de \LaTeX, pero esta plantilla dispone de comandos propios como por ejemplo el siguiente:

\imagen{escudoInfor}{Autómata para una expresión vacía}



\section{Listas de items}

Existen tres posibilidades:

\begin{itemize}
	\item primer item.
	\item segundo item.
\end{itemize}

\begin{enumerate}
	\item primer item.
	\item segundo item.
\end{enumerate}

\begin{description}
	\item[Primer item] más información sobre el primer item.
	\item[Segundo item] más información sobre el segundo item.
\end{description}
	
\begin{itemize}
\item 
\end{itemize}

\section{Tablas}

Igualmente se pueden usar los comandos específicos de \LaTeX o bien usar alguno de los comandos de la plantilla.

\tablaSmall{Herramientas y tecnologías utilizadas en cada parte del proyecto}{l c c c c}{herramientasportipodeuso}
{ \multicolumn{1}{l}{Herramientas} & App AngularJS & API REST & BD & Memoria \\}{ 
HTML5 & X & & &\\
CSS3 & X & & &\\
BOOTSTRAP & X & & &\\
JavaScript & X & & &\\
AngularJS & X & & &\\
Bower & X & & &\\
PHP & & X & &\\
Karma + Jasmine & X & & &\\
Slim framework & & X & &\\
Idiorm & & X & &\\
Composer & & X & &\\
JSON & X & X & &\\
PhpStorm & X & X & &\\
MySQL & & & X &\\
PhpMyAdmin & & & X &\\
Git + BitBucket & X & X & X & X\\
Mik\TeX{} & & & & X\\
\TeX{}Maker & & & & X\\
Astah & & & & X\\
Balsamiq Mockups & X & & &\\
VersionOne & X & X & X & X\\
} 

\capitulo{4}{Técnicas y herramientas}

Esta parte de la memoria tiene como objetivo presentar las técnicas metodológicas y las herramientas de desarrollo que se han utilizado para llevar a cabo el proyecto. Si se han estudiado diferentes alternativas de metodologías, herramientas, bibliotecas, se puede hacer un resumen de los aspectos más destacados de cada alternativa, incluyendo comparativas entre las distintas opciones y una justificación de las elecciones realizadas. 
No se pretende que este apartado se convierta en un capítulo de un libro dedicado a cada una de las alternativas, sino comentar los aspectos más destacados de cada opción, con un repaso somero a los fundamentos esenciales y referencias bibliográficas para que el lector pueda ampliar su conocimiento sobre el tema.


\section{Metodologías de Desarrollo}

\subsection{SCRUM}
\label{sub:scrum}

Scrum es un marco de desarrollo ágil que se caracteriza por realizar ciclos de desarrollo. Se basa en usar una estrategia incremental, frente al carácter más completo y planificado de las metodologías tradicionales, mediante iteraciones a las que denomina \textit{sprint}. 

En cada una de los sprint se revisa lo que se ha realizado durante la iteración anterior, y se determina que labores o tareas se pueden realizar en el nuevo ciclo. Esto aporta una gran cantidad de ventajas frente a los métodos tradicionales. \citep{wiki:SCRUM}.

\subsection{GitFlow}
\label{sub:gitflow}
El flujo de trabajo de Git permite establecer una forma organizada y ágil de llevar a cabo todos las aportaciones del proyecto. \citep{wiki:gitflow}.

Se basa en el uso de ramas para organizar el flujo de trabajo. La rama \textit{master} contiene el estado actual del desarrollo estable, el cual está listo para ser desplegado en cualquier momento. Se suele crear una rama cuando se quiere añadir una nueva característica, o realizar una nueva versión del programa. 
En el caso de este proyecto se ha mantenido la rama master y se ha creado una rama para las nuevas versiones. Estas han sido unidas a la rama master una vez listas para su despliegue.
Sin embargo, no se han creado ramas de la principal o de las versiones para corregir errores o incluir pequeñas modificaciones. 

Dado que no se cuenta con un equipo de desarrollo completo, el flujo habitual de trabajo se ha desarrollado sobre la rama principal, master, con ramas de desarrollo a modo de \textit{releases}. 

\subsection{Kanban}
\label{sub:Kanban}

Kanban es un sistema de organización y gestión de las tareas de un proyecto. \citep{wiki:Kanban}
Viene del japonés \textit{Kanban}, o \textit{tarjeta de señal} en Castellano, llamado de esta manera por el uso que se hace de tarjetas que describen las tareas a realizar. 

Su representación suele hacerse sobre una pizarra o panel en el que se van estableciendo las diferentes tareas a realizar en pequeñas tarjetas. Dichas tarjetas se organizan dentro del panel en diferentes zonas, como \textit{en progreso} o \textit{nueva tarea} o \textit{finalizada}. 
De esta forma es sencillo comprobar el estado del desarrollo de un vistazo. 

\section{Herramientas de gestión de repositorio}

\subsection{Git}

Git es una herramienta utilizada para llevar a cabo control de versiones. Es posiblemente la más utilizada entre empresas y desarrolladores, y viene integrada en la mayoría de los sistemas basados en UNIX.

Para llevar a cabo la gestión de las versiones, establece un repositorio del cual se guarda un histórico de todas las modificaciones que se han llevado a cabo, permitiendo establecer quien hizo que modificación, así como hacer uso del sistema de ramas mencionado en \ref{sub:gitflow}. 

\subsection{GitHub}

GitHub es un repositorio de código remoto. Es posiblemente el servicio de hospedaje de código más extendido entre empresas y desarrolladores independientes \citep{wiki:GitHub}.

Además de ser totalmente compatible con el sistema de ramas establecido en GitFlow \ref{sub:gitflow}, permite organizar el flujo de trabajo en forma de pequeñas tareas, que son fácilmente distribuibles en los diferentes sprints, ver \ref{sub:scrum}. 

De esta forma, unido a Git, se convierte en una herramienta de gran utilidad para llevar a cabo la correcta organización de un repositorio de código. Sobre todo si interviene más de un desarrollador. 

\subsubsection{GitLab}
En un principio, se valoró la posibilidad de hospedar un pequeño servidor GitLab que permitiese llevar a cabo el proyecto de forma privada de forma gratuita. Sin embargo, GitHub proporciona repositorios privados de forma gratuita a estudiantes, de forma que al final se eligió esta opción sobre GitLab.

\subsection{ZenHub}

ZenHub es una extensión que se integra con GitHub. Se trata de una implementación del método Kanban, \ref{sub:Kanban}. \citep{wiki:ZenHub}.

Muestra en el repositorio de GitHub, un panel informativo organizado por columnas que describen el estado de las tareas que en ellas se encuentran. 
Permite visualizar rápidamente el estado del Sprint en el que se está trabajando, así como de las tareas que han quedado en espera.

\subsection{GitKraken}

GitKraken es una aplicación de escritorio que permite hacer uso de Git desde una interfaz gráfica, \citep{wiki:GitKraken}. Presenta el flujo de trabajo que se ha ido realizando, las diferentes ramas y \emph{commits}, y permite realizar diferentes acciones relacionadas con Git, como \emph{pull-request}, \emph{commits}, creación y borrado de ramas, unión de ramas... etc.

Además provee de un sistema de resolución de conflictos entre archivos, que permite seleccionar los cambios a mantener de forma muy intuitiva y sencilla. 
La versión más reciente añade una implementación de Kanban que permite visualizar el mismo panel que muestra ZenHub en GitHub.

\subsubsection{GitHubDesktop}
Se valoró la posibilidad de utilizar GitHubDesktop, pero GitKraken es superior en algunos sentidos, ya que permite realizar ciertas tareas complejas con extrema facilidad, como deshacer errores o guardar cambios para más adelante (\textit{stash}), completamente integrado con GitFlow... etc.

\section{Herramientas de desarrollo}

\subsection{PyCharm}

PyCharm es un IDE, \textit{Integrated Development Environment}, para el lenguaje de programación Python desarrollado por JetBrains, \citep{wiki:PyCharm}. 

Se trata de un IDE que contiene todas las funcionalidades necesarias para el desarrollo de aplicaciones en Python, así como para la gestión de instalación de dependencias (vía \code{pip}, ver \citep{wiki:PyPa}, y la adquisición de documentación necesaria para el desarrollo. 

Es posiblemente el entorno de desarrollo más avanzado para Python. Se ha elegido porque permite hacer despliegue y \emph{debug} de aplicaciones en remoto, de forma que es de gran utilidad al realizar tareas de \emph{debug} en una RaspberryPi.

Cabe destacar que se está haciendo uso de una licencia de estudiante para acceder a todas las funcionalidades de PyCharm. Se puede obtener esta licencia a través de la página web, \citep{wiki:PyCharm}, creando una cuenta en la que se haga uso de un correo electrónico perteneciente a una institución educativa. 


\subsection{WebStorm}

WebStorm es un IDE, \textit{Integrated Development Environment}, para el desarrollo de aplicaciones web desarrollado por JetBrains, \citep{wiki:WebStorm}. 

Se trata de un IDE que contiene todas las funcionalidades necesarias para el desarrollo de aplicaciones en HTML, JavaScript, Node.js, Angular, Electron... etc, así como para la gestión de instalación de dependencias (vía npm o Yarn), y la adquisición de documentación necesaria para el desarrollo. 

Se trata de un entorno de desarrollo parecido a PyCharm, y su elección está un tanto condicionada por ello. Se ha utilizado para la programación del entorno web del que dispondrá el proyecto. 

\section{Herramientas de documentación}

\subsection{\LaTeX}

Para llevar a cabo la documentación del proyecto se ha hecho uso de \LaTeX{} \citep{wiki:latex}. Se trata de un lenguaje de marcas que permite la redacción de textos que presentan alta calidad tipográfica. La filosofía de trabajo con \LaTeX{} se basa en centrarse en el contenido y no en la forma. Es decir, el redactor de un documento no tiene porque centrarse en el formato, tipos de letra y demás, sino que su labor es redactar y por tanto se le deja dedicarse al contenido del mismo.

\subsubsection{TexMaker y TexLive}
TexMaker es un editor multiplataforma de \LaTeX. Se trata de un entorno parecido al habitual procesador de textos, que permite la redacción de textos haciendo uso del lenguaje de marcas \TeX. Se caracteriza por implementar un corrector, disponer de auto-completado de marcas para \LaTeX{} y un visor PDF del documento generado, \citep{wiki:TexMaker}.

TexLive es una distribución de \LaTeX. Se trata de un compendio de herramientas, fuentes y archivos de configuración que permiten la compilación de código \TeX{} a un documento legible, como un PDF, \citep{wiki:TexLive}.







\capitulo{5}{Aspectos relevantes del desarrollo del proyecto}
\label{cap:RelvAsp}
En este capítulo, se recogerán los aspectos que han determinado el desarrollo del proyecto. Tanto las ideas iniciales, como las finalmente implementadas, así como la motivación para tomar ciertas decisiones, y la manera en que se solucionaron los errores encontrados. 

\section{Inicio del proyecto}

La idea de este proyecto surgió al plantearme si quería pasar 12 créditos (300h) desarrollando algo que no me interesase ni motivase demasiado. Al fin y al cabo, 4 años de carrera deberían de acabar bien y, tal y como empezó, aprendiendo algo.

Así que se me ocurrió dar salida al drone que construí hace casi 4 años, haciéndolo formar parte de mi proyecto de final de carrera. En principio mi idea era que el drone recorriese un plano de un entorno abierto haciendo una ruta preestablecida. Grandes empresas del sector tecnológico (Amazon sobre todo) están jugueteando con la misma idea pero, generalmente por legislación, este proceso se está viendo muy lastrado. 

Me gustaba la idea. ¿Qué mejor forma de finalizar una carrera de ingeniería que hacer que un drone se estrelle contra un obstáculo a alta velocidad, entre un amalgama de hélices y piezas hechas añicos? Al fin y al cabo, las pruebas son pruebas, y algo se iba  a romper sí o sí. Mejor si era de una forma espectacular.

El \cotutorOne{}, uno de mis tutores, no parecía tan seguro, ya habían jugueteado con algo parecido en el pasado (en un simulador), y no le apetecía pasar de nuevo por lo mismo. Hasta que otro de ellos, el \cotutorTwo{}, mencionó la posibilidad de que el drone actuase como un vigilante nocturno en la universidad, haciendo uso de un algoritmo de localización sin GPS. 

Además, nos harían falta conocimientos fuertes sobre hardware y mecanismos de control, y un tercer tutor siempre viene bien, entra el \tutor{}.

Y así comencé con el desarrollo de este proyecto que, en algún momento, tendrá el potencial suficiente como para convertirse en el \emph{sereno} de la Escuela Politécnica Superior de la Universidad de Burgos.

\begin{figure}[H]
	\centering
	\includegraphics[width=0.5\textwidth]{Logo}
	\caption[Logo]{Logo de UBUDroneSereno.}\label{fig:Logo}
\end{figure}



\section{Metodología}

La gestión del proyecto se ha llevado a cabo a base de metodologías ágiles mediante \emph{Scrum}.
Sin embargo, cabe destacar que dado que el proyecto no se compone de varios integrantes, no es posible realizar en su totalidad el flujo de trabajo de Scrum, o este ha sido simplificado. 

\begin{itemize} 
\item Se han realizado iteraciones \emph{Sprint} bisemanales, en las que se han discutido las tareas a completar de cara a la siguiente reunión. 
\item Dichas tareas, denominadas \emph{Issues} han sido registradas haciendo uso de GitHub, que ha servido de plataforma de gestión de tareas y asignaciones. De nuevo, las asignaciones han sido realizadas por un único alumno. 
\item Se ha hecho uso de \emph{Kanban} mediante la implementación que ofrece ZenHub de un tablero para tarjetas, y de la herramienta Glo, disponible en GitKraken. 
\item El flujo de trabajo se ha distribuido mediante Kanban, en diferentes pilas (En progreso, Cerrada, Backlog, IceBox... etc)
\item Para comprobar el correcto ritmo de desarrollo del proyecto se ha hecho uso de los gráficos BurnDown que provee ZenHub, pero de nuevo, al no tratarse de un proyecto en el que intervienen múltiples desarrolladores, no es posible ajustarse al modelo presentado en los gráficos como ideal. 
\end{itemize}

Para el desarrollo de todas las partes del proyecto se ha seguido el mismo proceso: 
\begin{enumerate}
\item Se recaba toda la información necesaria para llevar a cabo la implementación.
\item Se estudia en profundidad la documentación.
\item Se comienza creando las clases de la implementación con una estrategia de diseño \emph{bottom-up}, aunque las clases de más alto nivel siguen una estrategia \emph{top-down} de forma que se requiere de una visión completa del proyecto para poder enlazarlas correctamente.
\end{enumerate}

No se ha seguido un diseño dirigido por pruebas, \emph{TDD}, ya que se ha considerado que genera una cantidad de código a refactorizar demasiado grande durante las etapas más tempranas, y dada la magnitud del proyecto solo conseguiría ralentizar el proceso de desarrollo. 
Se detallará más sobre las pruebas en la sección \ref{sec:testing}.


\section{Formación}

Para el desarrollo del proyecto se ha hecho uso de conocimientos adquiridos en la presente carrera, en muchos años de curiosidad y de búsqueda en internet. 

El desarrollo ha sido, en su mayor parte, escrito en Python, y se ha hecho uso de librerías como Numpy, Matplotlib y SciKit que se estudian durante la carrera, y otras que no, como PySerial (para establecer una comunicación serie), Bluetin-Echo (para hacer uso de los sensores de distancia HC-SR04), el framework Flask (para el desarrollo de la web-app) y la API WebRTC (para el vídeo en tiempo real) entre otras.

Sin embargo, cabe destacar la lectura de ciertos artículos, y el acceso a ciertos recursos:

\begin{itemize}
\item \textit{Artificial Intelligence for Robotics}. Un curso de Udacity impartido por Sebastian Thurn, disponible en \citep{wiki:UdCityPF}. En él se explican componentes habituales en múltiples proyectos, como el controlador PID, el filtro de Kalman, y otros no tan habituales, como el FIltro de Partículas utilizado en este proyecto para lograr un sistema de localización que no dependa de GPS, ver sección \ref{subsec:PF}.
\item MultiWiiSerialProtocol. Interfaz o definición del protocolo de comunicación MultiWii, utilizado en prácticamente todas las controladoras de vuelo actuales, como base de comunicación con ordenadores. Este protocolo es el utilizado para comunicar la RaspberryPi con la controladora de vuelo, y enviar/recibir comandos. Disponible en \citep{wiki:MSPDefinition}.
\item \emph{Getting Started with WebRTC}, un compendio de las funcionalidades que ofrece WebRTC para lograr comunicación en tiempo real. Utilizado para lograr vídeo sin apenas latencia entre un navegador y la RaspberryPi a bordo del drone. Disponible en \citep{wiki:WebRTCFullDesc}.
\item \emph{The Vector Field Histogram - Fast Obstacle Avoidance for Mobile Robots}. Se trata del artículo original de Borenstein y Koren, en el que detallan el funcionamiento del sistema de evasión de obstáculos implementado. Disponible en \citep{art:BorensteinKorenVFH}.
\item \emph{The Flask Megatutorial}. Miguel Grinberg da una clase magistral sobre como empezar, continuar y finalizar con Flask. Este mega-tutorial contiene todo lo necesario para preparar una web-app con Flask. Disponible en \citep{wiki:Flask}.
\end{itemize}

Por supuesto cabe destacar la ingente documentación disponible de Python, y todas las librerías que lo componen.
Sin duda hay muchas otras fuentes de formación relacionada con este proyecto, y pueden ser consultadas en la bibliografía. 



\section{Desarrollo de algoritmos}

En el desarrollo de este proyecto, se ha llevado a cabo el desarrollo de tres algoritmos fundamentales: 

\begin{itemize}

\item Un sistema de evasión de obstáculos. Llevado a cabo mediante la implementación de dos algoritmos: el algoritmo de Campos Potenciales, y el algoritmo VFH en sustitución del primero.
\item Un sistema de localización sin depender de la red de satélites GPS/Glonass. Llevado a cabo mediante la implementación de un Filtro de Partículas basado en simulación de Montecarlo, para estimar la variable \emph{posición}, en base a las variables \emph{distancias a obstáculos}.
\item Varios controladores automatizados para complementar la evasión de obstáculos, y lograr la autonomía del sistema. Llevados a cabo mediante implementaciones de controladores PID, alimentados por la salida del algoritmo VFH, y por las lecturas de los sensores de la controladora de vuelo (para la inclinación) y un sensor de ultrasonidos colocado para medir la altura del drone.
\end{itemize}

\subsection{Potential Fields}

Durante el primer Sprint, y mientras decidíamos que algoritmos deberíamos usar para lograr el propósito del proyecto, el \tutor{} sugirió la utilización del algoritmo de Campos Potenciales para llevar a cabo la evasión de obstáculos.

El algoritmo de campos potenciales se basa en la idea de que existe una meta a la que se quiere llegar, y una serie de obstáculos. 

La meta genera una fuerza de atracción sobre el drone, y los obstáculos fuerzas repulsoras. 
En primera instancia este algoritmo es relativamente sencillo de implementar, y no tiene un coste computacional muy alto. Toda la dimensionalidad del problema se acaba reduciendo a una suma de fuerzas, que determinan el vector resultante, con una dirección y un sentido que el drone deberá seguir. 

Sin embargo, este método presenta ciertos inconvenientes:
\begin{itemize}
\item Si el drone entra en una zona con forma de `U', podría atascarse en un mínimo local, entrando y saliendo de ella indefinidamente. 
\item Una zona estrecha es todo un desafío. Las paredes de un pasillo, por ejemplo, presentan fuerzas repulsoras sobre el drone, y la fuerza atrayente de la meta no es suficiente como para mantener la dirección del drone a través del pasillo, con lo que este empezará a oscilar (si es que ha conseguido entrar en él) o ni siquiera será capaz de pasar de la entrada del pasillo.
\end{itemize}

Por ello se ha hecho uso del Vector Field Histogram.

\subsection{Vector Field Histogram}
\label{subsec:VFHcomments}
El Vector Field Histogram se basa en el uso de una malla de certidumbre. Dicha malla es una representación del plano, el cual se divide en celdas (muy apropiado para usarlo con Numpy, y aprovechar la potencia del cálculo vectorizado de que provee). 

Cada celda contiene un valor, dicho valor es la certeza de que en ella exista un obstáculo. Por tanto se define un tamaño para las celdas (5x5 cm es una buena medida teniendo en cuenta el rango de detección de nuestros sensores de ultrasonidos), y se realizan medidas mediante los sensores de ultrasonidos. 

Encontrar un obstáculo en cierta posición supone incrementar en 1 la cuenta que existe en la celda que representa esa distancia del agente. 
Computacionalmente, este método supera al algoritmo de Campos Potenciales, ya que no es necesario tener en cuenta en que posición del rayo que dispara el sensor (recordemos de la documentación teórica que el HC-SR04 tiene una amplitud de unos 30º), creando una distribución Gaussiana alrededor de la bisectriz del rayo, sino que únicamente se incrementa la posición central. 

Puede parecer una sobresimplificación del problema, pero lo cierto es que tras múltiples medidas de los sensores disponibles, las malla de certidumbre representa con bastante fidelidad el estado del entorno cercano al drone. 

Una vez obtenida la malla, esta se convierte en un histograma polar. Para ello se divide esta en sectores de una amplitud determinada (5º se ha mostrado como una cantidad aceptable, generando así 72 sectores que cubrirán los 360º). De esta forma, ya se dispone de una representación aproximada de la ubicación de los obstáculos. 

Aún así se somete el histograma a una función de suavizado. En el artículo presentado por Borenstein y Koren en \citep{art:BorensteinKorenVFH}, la función utilizada contiene una errata. Tras discutirlo con mis tutores, decidimos utilizar cualquiera de las funciones de suavizado o filtrado disponibles en SciKit, y me decanté por la señal de Hann. 
De esta forma, los obstáculos parecen \textit{estirarse}, ya que la señal de Hann permite establecer la simetría de los valores a suavizar. Así que la certidumbre sobre la existencia de obstáculos en un punto contiguo a los obtenidos en la malla de certidumbre se verá incrementada, creando un histograma todavía más conservador sobre la posición de los obstáculos. 

Por si esto pudiera parecer poco, se define un umbral. El umbral establece cual es el nivel máximo de certidumbre de obstáculos en una zona, como para considerarla segura para navegar. 
El histograma presenta una serie de subidas y bajadas, como valles y colinas. Los valles, son precisamente las zonas que, umbral mediante, serán seguras para navegar.

Una vez obtenidos los valles (compuestos de varios sectores seguros), el algoritmo establecerá cual es el sector elegido para dirigirse a él, en base a la ubicación de la meta a la que se quiere llegar. 
Precisamente en este paso se determina la fortaleza de este método frente a los Campos Potenciales. El VFH va a elegir siempre el centro del valle para navegar, de forma que puede navegar en zonas estrechas, siempre que el umbral lo permita. 

Y precisamente en este paso es donde se encuentra otro problema en el artículo de Borenstein y Koren en \citep{art:BorensteinKorenVFH}, si siempre se elige el sector central para navegar, en el momento en el que se sale de una región estrecha (un pasillo por ejemplo) y la meta no está en el centro, el drone empezará a ejecutar un movimiento circular aproximándose a la meta en un movimiento cada vez más cerrado pero, matemáticamente, puede que infinito. Por ello se incluyó una modificación en el algoritmo original, que permite aproximarse al destino de forma directa, si este se encuentra a la vista del drone, esto es, en un sector seguro para su navegación.

La salida del algoritmo se ha codificado como la orientación que el drone debería seguir, es decir, presenta un valor en el intervalo $[-179, 180]$, el cual expresa cuantos grados debería rotar el drone, y dado el signo la dirección de rotación, para dirigirse a la meta a través de un sector seguro. 
Además, el VFH, proporciona una medida de la velocidad que debería alcanzar el drone, en base a como de ocupado está el sector en el que el drone navega. La velocidad será proporcionada en base a la certidumbre de obstáculos, de manera que se irá reduciendo para poder realizar un cambio de sentido.

Este valor, por si solo, no es de mucha utilidad ya que no es interpretable por la controladora de vuelo, sino que tiene que ser \textit{convertido} a un valor entre $[1000, 2000]\mu s$. Para ello, se ha hecho uso de controladores PID, ver \ref{subsec:PIDcomments}.

\subsection{Particle Filter}
\label{subsec:PFcomments}
El Filtro de Partículas es una pieza clave en el funcionamiento del proyecto. Provee de información sobre la posición del agente de forma muy robusta, ya que está basado en un modelo probabilístico que obtiene sus valores mediante un método de Montecarlo. 

Básicamente, el algoritmo crea \emph{drones virtuales} (partículas) que obtienen medidas hasta los obstáculos. Conociendo esas medidas, y comparándolas con las del drone real, parece inmediato poder decir cuanto se parece un \emph{drone virtual} al drone real, y establecer esa posición como la posición del drone. 

Y por supuesto, en teoría casi todo es sencillo: Los diferentes cursos, artículos, páginas web y tutoriales basados en teoría, siempre hablan de \emph{landmarks}. Estos \emph{landmarks}, son zonas reconocibles por el agente (una diana en una pared, una marca... etc), predispuestas sobre el entorno, de manera que tanto las partículas como el agente pueden medir sus distancias a ellos, y de esta forma calcular como de parecidas son las partículas al agente. 

Sin embargo, a la hora de la práctica, no hay tales \emph{landmarks}. Por el sencillo hecho de que si tu agente no tiene al alcance alguno de esos \emph{landmarks}, no es nada probable que pueda medir su distancia a este. Pero sí tenemos paredes. 

Así que lo que se hace es calcular las distancias teóricas de cada una de las partículas a los obstáculos dentro de su área de acción (basada en la distancia máxima alcanzable por los sensores HC-SR04), y desear que no haya zonas simétricas ya que adquirirían la misma probabilidad de ser la zona en la que se encuentra el drone.

El drone por su parte toma medidas de las distancias, y se comparan las medidas reales con las de todas las partículas. 

El drone se desplaza, y las partículas con él. Cada vez que el drone hace un movimiento las partículas deben ejecutar ese mismo movimiento. Sencillo. Sí, en teoría. El drone no gira o se desplaza la cantidad de grados o centímetros que se le ordena en cuanto se le ordena, el giro o el desplazamiento lleva tiempo. ¿Cuánto tiempo? No es nada sencillo de aproximar, depende del estado de la batería, de la temperatura (los ESC del drone pueden llegar a reducir la velocidad de giro del motor si notan demasiado calor, aunque no es habitual), del cálculo de los controladores PID... etc. 
Es decir, existe una incertidumbre MUY grande, sobre todo en el desplazamiento. El movimiento de rotación es más sencillo de aproximar dado que se dispone de un magnetómetro, aunque tampoco es fiable al 100\% ya que puede estar alterado por la desviación magnética, diferente en cada punto del planeta, por la presencia de fuentes de magnetismo (motores, 4 en este caso), o fuentes de corriente elevadas (una batería alimentando 4 motores, por ejemplo).

De manera que se incluye cierta incertidumbre en cualquier movimiento realizado por las partículas, de esta forma la población de partículas va divergiendo, y alejándose poco a poco de la posición exacta del drone, en su mayoría. 

Debido a ello, la población cada vez es menos útil al propósito de localización, así que se seleccionan individuos en base a la probabilidad de ser la posición real del agente. 
La selección se realiza con remplazo, es decir, una misma partícula puede aparecer varias veces. De esta forma la población vuelve a converger. 

Y con esto ya se dispondría de una ubicación aproximada para el drone. Conocer la posición es fundamental para el funcionamiento del sistema de evasión de obstáculos, ver subsección \ref{subsec:VFHcomments}, dado que el VFH trata de guiar el drone hacia la meta en una ruta segura, ¿cómo sabe hacia donde debe dirigir el drone, si no sabe dónde se encuentra?

De esto se puede extraer una lección de lo más filosófica: Primero descubre donde estas, para luego poder dirigirte hacia tu meta.


\subsection{PID}
\label{subsec:PIDcomments}
Para llevar a cabo el control automatizado del drone, no es suficiente con proporcionar la salida de los diferentes algoritmos a la controladora de vuelo. 

La controladora de vuelo entiende entradas en sus canales (Acelerador, Inclinación, Balanceo, Rotación, Armar/Desarmar) en el intervalo $[1000, 2000]\mu s$, de manera que recibiendo la información proveniente de algunos sensores, se han creado los siguientes controladores PID: 

\begin{itemize}
\item Control de Altitud: Es fundamental disponer de un mecanismo de control constante de la altitud. Se alimenta de la información proveniente de un sensor de ultrasonidos que mira hacia el suelo, de forma que se puede establecer la altitud que debe alcanzar el drone, relativa al suelo sobre el que se encuentra. Se encarga del canal 1, o Acelerador.
\item Control de Orientación: Este controlador se encarga de dirigir al drone por el entorno, mediante cambios en su orientación. Se alimenta de la salida del VFH, ver subsección \ref{subsec:VFHcomments}, para establecer su objetivo, y del magnetómetro para establecer la medida real, y así crear una salida entre $[1400, 1600]\mu s$.
Esta salida es debida a la posición central del canal, $1500\mu s$, estableciendo la \emph{no} rotación del drone en ningún sentido. Se encarga del canal 4, o Rotación.
\item Control de Velocidad: Este controlador se encarga de controlar la inclinación del drone. No es posible determinar una velocidad concreta para el desplazamiento, dado que es muy variable, pero si se puede establecer la inclinación que debe alcanzar el drone. Esta inclinación es generada en base a los obstáculos en el sector en que se navega, y provista por el VFH, ver subsección \ref{subsec:VFHcomments}. De forma que tanto el objetivo, como el ajuste a realizar están automatizados. Genera una salida entre $[1400, 1600]\mu s$. La medida actual de la inclinación puede obtenerse de la controladora de vuelo. Se encarga del canal 2, o Elevación.
\end{itemize}


\section{Comunicaciones}

En el desarrollo de este proyecto, se ha llevado a cabo el desarrollo de tres sistemas de comunicación: 

\begin{itemize}
\item Un sistema de comunicación entre un ordenador y un Drone. Llevado a cabo mediante la librería de implementación propia \emph{MSPio}.
\item Un sistema de control remoto de un drone a través de un navegador web. Llevado a cabo mediante Flask, Python y la librería MSPio.
\item Un sistema de vídeo en tiempo real. Llevado a cabo mediante la API de WebRTC, HTML, JavaScript, y en la parte del servidor (la RaspberryPi) se ha hecho uso de UV4L, un driver que accede a la cámara de la Raspberry, y provee de un servidor WebRTC al que conectarse.
\end{itemize}

\subsubsection{MSPio}

En este proyecto, la RaspberryPi no consta de un sistema de tiempo real, por lo tanto no es lo ideal para llevar a cabo el control de un sistema de tiempo real como lo es un drone. De esta manera, se ha relegado esta tarea en una controladora de vuelo. 

Las controladoras de vuelo actuales, en su mayoría, implementan un mecanismo de comunicación con un ordenador para su configuración. 
Dicha configuración se realiza a través de una aplicación que muestra toda la parametrización que puede ser llevada a cabo (parámetros de PID, canales de radio, protocolos de radio, telemetría, alarmas, calibración... etc)

El protocolo que utilizan dichas controladoras de vuelo es generalmente el mismo: MultiWii Serial Protocol. 
Este protocolo se hereda de las primeras versiones de controladoras de vuelo, basadas en los mandos de la consola Wii de Nintendo\footnote{Wii y Nintendo son marcas registradas de Nintendo Co. LTD}, y de ahí su nombre.
La definición del protocolo es pública, así que cualquiera puede crear una implementación. 

Para este proyecto, se ha creado una implementación para Python 3.x, a la que se ha llamado MSPio.
Esta implementación es totalmente genérica, y permite el envío de órdenes a la controladora de vuelo, entre ellas la entrada de los canales que permiten mover el drone, así como la solicitud de información, como inclinación, estado de la batería, temperatura... etc.

Para mayor comodidad, se han creado varios métodos que permiten la obtención de información , o el envío de los canales de radio como una lista de valores, con tan solo llamarlos. Aún así, se ha determinado como \emph{genérico}, ya que se ha definido el parseo de los valores de respuesta de la controladora de vuelo, como una cadena de texto.
De forma que implementar la adquisición de nuevos datos es tan sencillo como pasar como parámetro una cadena de texto que representa la estructura de retorno. Ver tabla \ref{tb:MSP_MESSAGES}

\subsubsection{Control Remoto}

El mecanismo de control remoto se ha llevado a cabo mediante un navegador web, dado que proporciona un acceso sencillo y sin barreras a cualquier usuario. 

Tan sencillo como iniciar sesión, el sistema establece que drone tiene asignado el usuario, y le conecta con él. La interfaz web permite la visualización en tiempo real del feed de vídeo disponible en el drone, a traves de WebRTC, y su control con tan solo conectar una emisora al puerto USB del ordenador. 

Una vez conectada la emisora, la web reconoce la misma como un joystick, y es capaz de leer los canales, y enviarlos al servidor remoto, el cual los convierte en valores aceptables para el drone, y se los envía a este a través de una comunicación socket.
Para activar el control remoto basta con activarlo y el drone ya estará listo para ser controlado.

De esta manera, el usuario final no tiene una comunicación directa con el drone, sino que el servidor intermedio es quien se encarga de esta tarea. 

Esto libera a la RaspberryPi de hospedar un servidor web y recibir clientes, así como del uso de un canal de comunicación potencialmente no seguro entre el cliente y el drone. 


\begin{figure}[H]
	\centering
	\includegraphics[width=0.9\textwidth]{webMain}
	\caption[Web index]{Página principal de la aplicación.}\label{fig:webmain}
\end{figure}

\subsubsection{Vídeo en Tiempo Real}

Embeber vídeo en una página web puede parecer trivial, pero no lo es en absoluto. Los contenedores existentes en HTML requieren que el vídeo esté finalizado para poder reproducirlo (MP4/WebM deben definir la duración del vídeo), lo cual ha dado varios quebraderos de cabeza. 

Se barajaron múltiples posibilidades, entre ellas:
\begin{itemize}
\item Hacer uso de MJPEG (Motion JPEG) consistente en tomar gran cantidad de fotografías e ir actualizando un canvas con cada nuevo fotograma. Sencillo, y simple de implementar, pero de un coste computacional inasumible dada la cantidad de trabajo que debe desempeñar la RaspberryPi.
\item Hacer uso de MPEG-DASH (Dynamic Adaptive Streaming over HTTP), pero se mostró como una solución con un coste computacional elevado, dado que suponía instalar el servidor en la RaspberryPi, hacer que este realizase streaming y que el servidor principal actuase como un repetidor para los usuarios.  
\end{itemize} 

Finalmente nos decantamos por WebRTC, dado que permite establecer una comunicación directa entre dos navegadores, sin necesidad de hacer uso de HTTP. El problema que plantea es que se trata de una API para trabajar entre navegadores, y no es elegante ni fiable mantener una navegador constantemente abierto en la RaspberryPi, dado que requiere de intervención del usuario para permitir acceso a la cámara. 
La solución pasaba por utilizar, o implementar, un servidor WebRTC, que implementase el protocolo \emph{Session Description Protocol} para comunicar un navegador cliente con la RaspberryPi. Por no añadir más carga de trabajo se ha hecho uso de UV4L, el cual es precisamente eso, un servidor que implementa las funcionalidades necesarias para hacer uso de WebRTC.

\section{Plataforma de usuario}

El manejo del drone por parte del usuario se ha diseñado para ser de tremenda simplicidad. 
Tan solo se muestra una página de login, tal y como se muestra en la figura \ref{fig:login}, que permite al usuario iniciar sesión. 

\begin{figure}[H]
	\centering
	\includegraphics[width=0.7\textwidth]{loginWeb}
	\caption[Página de Login]{Página de inicio de sesión para los usuarios.}\label{fig:login}
\end{figure}

Una vez registrado en el sistema, se le muestra una interfaz que permite iniciar o detener el streaming de vídeo por parte del drone. De esta forma, la RaspberryPi solo activa el servidor de vídeo cuando este se está visualizando, tal y como puede verse en la figura \ref{fig:vidStream}. 
\begin{figure}[H]
	\centering
	\includegraphics[width=0.7\textwidth]{vidStream}
	\caption[Página de control]{Página de del drone asignado para los usuarios.Los LED IR están activos dada la condición de baja luminosidad.}\label{fig:vidStream}
\end{figure}

De conectar una emisora al equipo mientras el navegador está seleccionado, se reconoce esta y permite controlar el drone con tan solo pulsar un botón, tal y como puede verse en la figura \ref{fig:fullConn}.

\begin{figure}[H]
	\centering
	\includegraphics[width=0.7\textwidth]{fullConn}
	\caption[Página de control y emisora]{Página del drone asignado para los usuarios lista para activar el control.}\label{fig:fullConn}
\end{figure}

La interfaz ha sido desarrollada en HTML5 y JavaScript, y la aplicación web se basa en el framework Flask, con la intención de aprender un nuevo medio de desarrollo de aplicaciones, y continuar trabajando con Python. 



\section{Hardware empleado}

El equipamiento empleado en el drone es el siguiente: 

\begin{itemize}
\item Un drone de 450mm de diagonal, distancia medida de motor a motor. Equipado con motores de 810KV (810rpm/V), controladores de velocidad (ESC) de 20A, una controladora de vuelo Flip32 con acelerómetro, giróscopo, magnetómetro y barómetro. Las hélices empleadas son de 9\" de longitud y 4.5\" de paso, generando un empuje máximo de 3.8Kg.
\item Una batería LiPo de 3700mAh y 35C (\emph{C} es la capacidad de descarga de la batería, en este caso sería $35 * 3700mAh = 129500mAh = 129.5A$ máximo).
\item Seis sensores de ultrasonido HC-SR04. Cinco para las distancias a obstáculos y uno para controlar la altitud.
\item Una RaspberryPi 3B.
\item Una cámara PiNoIR. Sin filtro de infrarrojos, lo que permite el uso de LEDs infrarrojos para visión nocturna.
\item Dos anillos de LED infrarrojos para visión nocturna.
\item Seis divisores de voltaje para reducir el voltaje de salida de los sensores, ya que la RaspberryPi funciona a 3.3V y los sensores a 5V.
\item Cables para conectar todo.
\end{itemize}

\begin{figure}[H]
	\centering
	\includegraphics[width=0.7\textwidth]{SR04layout_bb}
	\caption[Conceptual de conexión de sensores a RaspberryPi]{Vista conceptual del conexionado de los sensores a la RaspberryPi.}\label{fig:schHCPi}
\end{figure}

\begin{figure}[H]
	\centering
	\includegraphics[width=0.7\textwidth]{SR04layout_schem}
	\caption[Diagrama de conexión de sensores a RaspberryPi]{Vista esquemática del conexionado de los sensores a la RaspberryPi.}\label{fig:concepHCPi}
\end{figure}

\begin{figure}[H]
	\centering
	\includegraphics[width=0.75\textwidth]{voltDivider}
	\caption[Divisor de voltaje]{Vista imagen de los divisores de voltaje creados.}\label{fig:imgVoltDiv}
\end{figure}

\begin{figure}[H]
	\centering
	\includegraphics[width=0.6\textwidth]{IRLED}
	\caption[LED IR]{Vista esquemática del conexionado de los LED Infrarrojos.}\label{fig:schIRLED}
\end{figure}

\begin{figure}[H]
	\centering
	\includegraphics[width=0.9\textwidth]{droneSide}
	\caption[Lateral del drone]{Detalle del drone completo. Lateral.}\label{fig:droneSideView}
\end{figure}

\begin{figure}[H]
	\centering
	\includegraphics[width=0.9\textwidth]{droneFront}
	\caption[Frontal del drone]{Detalle del drone completo. Frontal.}\label{fig:droneFrontView}
\end{figure}


\section{Testing}

Sin duda la parte más relevante del proyecto es la implementación física del mismo. Las pruebas llevadas a cabo se han centrado en la componente práctica, y en pruebas de campo. 

Las diferentes clases desarrolladas han sido testeadas de forma interna, y no se dispone de pruebas como tal, que puedan ser desplegadas en un sistema de integración continua. 
La motivación de no realizar este tipo de pruebas es la siguiente: 
\begin{itemize}
\item Las pruebas relacionadas con la comunicación entre el drone y la RaspberryPi requieren de una controladora de vuelo, y hacer un mock de este componente, supondría tener que estudiar la implementación completa que ofrece una controladora de vuelo al protocolo MultiWiiSerialProtocol, así como de todos los sensores. De no hacerlo, no existen sistemas de integración que pongan a nuestra disposición una controladora, un equipo que abra un puerto serie y que ejecute una serie de comandos/solicitudes interpretables por dicha controladora. Se incluye una pequeña clase de prueba que permite comprobar si se adquieren correctamente los datos de la controladora de vuelo, dibujando un plot de los ejes de inclinación y orientación, tal y como puede verse en la figura \ref{fig:mspIMU}, y en el vídeo \href{https://universidaddeburgos-my.sharepoint.com/:v:/g/personal/mbm0089_alu_ubu_es/EezmUjB1BSdKp-_VQrUkIXwBPSrvQdSnmhSTd-QA3jJaIQ?e=RYEaIl}{V\#1}.
\begin{figure}[H]
	\centering
	\includegraphics[width=0.9\textwidth]{mspIMU}
	\caption[Recepción de la IMU mediante MSPio]{Salida de los sensores de la controladora de vuelo. En azul la elevación, en naranja el balanceo y en verde la orientación.}\label{fig:mspIMU}
\end{figure}


\item De igual manera sucede con el sistema de control Remoto, que depende de la controladora de vuelo para comprobar su completo funcionamiento. De testearlo de forma parcial, solo se conseguiría comprobar que los paquetes llegan de forma correcta hasta el destino, y esta prueba se ha realizado como parte de las pruebas de campo, ver subsección \ref{subsec:fieldTestingv1}, de forma satisfactoria, ya que ayudó a corregir un bug de difícil detección.
\item Las pruebas relacionadas con el sistema de evasión de obstáculos, requieren de los diferentes sensores de distancia, de nuevo no existe ningún sistema de integración que proporcione dicho hardware. Realizar un mock de dichos sensores supone perder por completo la dinámica del sistema, i.e, los sensores no son perfectos, dan medidas con ruido, el drone no siempre se moverá con perfección milimétrica, los sensores no están perfectamente alineados... etc. Se realizaron pruebas durante su desarrollo mediante mock de los sensores, la cual puede ejecutarse si se utiliza la clase del VFH como \emph{\_\_main\_\_}.
\item Las pruebas relacionadas con el sistema de localización están basadas en los mismos sensores anteriores. El mock de dichos sensores para esta clase caería en una sobresimplificación del sistema, tal y como ocurre en el curso de UdaCity, ver \citep{wiki:UdCityPF}, que solo daría una sensación de falsa funcionalidad del sistema. Realizar pruebas sobre el funcionamiento base del Filtro de Partículas no es relevante, ya que no existe la dinámica encontrada en un entorno real. Pese a ello, en la clase que representa el Filtro de Partículas, se encuentra una pequeña prueba de su funcionamiento para un pequeño mapa en el que existen tres obstáculos. La dimensionalidad que alcanza el filtro es tal, que durante las pruebas no es nada sencillo apreciar el recorrido del algoritmo. De igual manera existe una pequeña prueba en la clase que representa los cálculos geométricos requeridos para el Filtro de Partículas.
\end{itemize}


\subsection{Pruebas de Campo}

Las pruebas de campo han permitido demostrar el funcionamiento del drone, así como encontrar errores en la programación. Se han mostrado de gran utilidad no solo en la búsqueda de bugs, sino como fuente de información para futuras mejoras así como para proyectos derivados o complementarios a este.

Esta sección se ha dividido en dos subsecciones, cada una establecida en base a la release a la que hacen referencia las pruebas. 


\subsubsection{Primera Release: v0.1.0-alpha}
\label{subsec:fieldTestingv1}

Las primeras pruebas de campo llevadas a cabo, se realizaron en el polígono industrial de Villalonquejar, en Burgos. 
Dado que la primera release no incluye ningún elemento de automatización, sino que comprende el apartado relacionado con el control remoto del drone, pudieron ser llevadas a cabo sin demasiados inconvenientes en una zona apartada, pero no cerrada. 

\begin{figure}[H]
	\centering
	\includegraphics[width=0.9\textwidth]{ftest1}
	\caption[Field Test 1. Villalonquejar]{Drone despegando durante el primer field test.}\label{fig:ftest1Drone}
\end{figure}

La comunicación con el drone se estableció creando un punto de acceso desde un ordenador portátil. Seguidamente, se inició el servidor web en el portátil, el sistema de control del drone, y servidor WebRTC ambos en la RaspberryPi. El resultado de las pruebas puede verse en el vídeo \href{https://universidaddeburgos-my.sharepoint.com/:v:/g/personal/mbm0089_alu_ubu_es/ERit03PQ4GVJvVXNLCxQQwUBUjZt6VjCwl5GcLUYwFQGPQ?e=dD258g}{V\#3}.

Las pruebas comprenden los siguientes apartados, puntuados según la calidad del resultado en una escala de $(0, 10]$: 
\begin{itemize}
\item Comprobación del correcto despliegue del sistema en una situación real: El despliegue se efectúa sin problema, la RaspberryPi responde correctamente a la comunicación vía SSH. Se realiza el despliegue perfectamente haciendo uso de la herramienta de \emph{deployment} de que provee el IDE PyCharm. 10/10.
\item Comprobación del sistema de vídeo en tiempo real: La comunicación con el servidor WebRTC es perfecta en este sentido. No existen retardos alarmantes en el feed de vídeo con una resolución de $1280x720$ a $15f/s$. Reducir la resolución y aumentar los fps podría ser una solución mejor, aunque el sistema no se moverá a una velocidad alta. La calidad del vídeo, sin embargo, es cuestionable. Al tratarse de una cámara sin filtro de luz infrarroja, los colores parecen algo saturados en ocasiones.  9/10.
\item Comprobación del sistema de login, web y comunicación con el drone: El login se realiza de forma correcta, se reconoce el drone asignado al usuario y se activa la comunicación con este sin problema. 10/10.
\item Comprobación del sistema de control remoto en tiempo real: El sistema de control remoto se activa sin problema, la comunicación es relativamente fluida, el drone responde correctamente a los comandos enviados, aunque con cierta latencia. Cabe destacar que los mensajes viajan hasta el servidor web que posteriormente se encarga de enviarlos al drone. 
Se encuentra un bug, referenciado en la issue \href{https://github.com/mbm0089/GII_0_17.02_SNSI/issues/53}{\#53}, existente a la hora de parsear el paquete JSON recibido por el servidor de control remoto existente en el drone. La recreación del bug, puede verse en el vídeo \href{https://universidaddeburgos-my.sharepoint.com/:v:/g/personal/mbm0089_alu_ubu_es/ERit03PQ4GVJvVXNLCxQQwUBUjZt6VjCwl5GcLUYwFQGPQ?e=dD258g}{V\#3} a partir del minuto 3:32. 
El bug detectado se debía a la debilidad de la señal al alejarse el drone del punto de acceso, ocasionando la pérdida de paquetes.
\end{itemize}

La realización de las pruebas ha dado como resultado la valoración de la release \emph{v0.1.0-alpha} como lista para ser publicada. Sin embargo se mantiene como versión \emph{alpha} dada la necesidad de realizar todavía más pruebas. 
Gracias al bug detectado, y solucionado, se valora incluir en la RaspberryPi una antena wifi mejor que la existente, o proporcionar a la RaspberryPi un modem 4G conectado vía USB para mantener siempre comunicación con el servidor. De esta forma se reducen las probabilidades de perder paquetes. 


\subsubsection{Segunda Release: v0.2.0-alpha}
\label{subsec:fieldTestingv2}

La segunda release del proyecto, traerá las siguientes mejoras y características sobre la anterior versión:
\begin{itemize}
\item Grabación del feed de vídeo en el ordenador cliente, haciendo uso de WebRTC.
\item Inclusión de diferentes sistemas de control automatizado, establecidos mediante un sistema de prioridades.
\item Automatización del control de altitud.
\item Automatización del control de velocidad.
\item Automatización del control de dirección.
\item Localización sin GPS.
\item Evasión de obstáculos.
\item Seguimiento de rutas.
\end{itemize}

Las pruebas realizadas comprenden los siguientes apartados, puntuados según la calidad del resultado en una escala de $(0, 10]$. Dada la peligrosidad de algunas de las pruebas, se ha llevado el drone a una nave industrial, ver agradecimientos en \ref{sec:agradecimientos}, dado que provee de un entorno seguro y cerrado en el que probar mecanismos automatizados:
\begin{figure}[H]
	\centering
	\includegraphics[width=0.7\textwidth]{ftest2init}
	\caption[Field Test 2. Nave industrial]{Drone sobre superficie blanda durante el segundo field test}\label{fig:ftest2init}
\end{figure}

\begin{itemize}
\item Calibrado del sistema de control de altitud: Se trata de un controlador PID que debe ser parametrizado. Dada la dinámica del sistema, tan dependiente del estado de la batería, de los cambios de contexto realizados por el sistema (recordemos que no es un RTOS\footnote{Real Time Operative System}), de los errores de medida proporcionados por el sensor... etc, la parametrización del PID ha sido bastante compleja. De hecho durante el transcurso de las pruebas, se rompió una hélice y una de las patas que hacían de apoyo para el drone, dificultando esto último, todavía más, el proceso de estabilización durante el despegue. 
\begin{figure}[H]
	\centering
	\includegraphics[width=0.7\textwidth]{ftest2cripple}
	\caption[Field Test 2. Demasiada kP]{Drone perdiendo una pata.}\label{fig:ftest2cripple}
\end{figure}
\begin{figure}[H]
	\centering
	\includegraphics[width=0.7\textwidth]{ftest2newleg}
	\caption[Field Test 2. Prótesis]{Drone con prótesis.}\label{fig:ftest2newleg}
\end{figure}

La motivación para retirar el material blando que puede verse en la figura \ref{fig:ftest2init}, es que es tan absorbente que falseaba las medidas tomadas por el sensor de ultrasonidos.

En el vídeo \href{FALTA REF!!!}{V\#4} puede verse el proceso llevado a cabo. 

Se detectó un bug existente en el cálculo del PID que llevaba a la pérdida de precisión al realizar una conversión a entero, antes de realizar redondeo. Esto podía causar que el error acumulado se disparase, y por lo tanto dar a la componente Integral del sistema una importancia no deseada. Dado que se trata de un bug menor, se solucionó \textit{in situ} por lo que no se generó una issue para ello.

Se detectó un bug existente en el sistema de control que no permitía que un controlador automatizado controlase cuando armar el drone. Esto produciría que el sistema de control de altitud, no pudiese actuar. Se trata de un bug menor, basado en añadir una pequeña espera en la que se mantiene la entrada del canal 5 (armar/desarmar) a un valor mínimo durante unos segundos, a la espera de que el sistema calibre el acelerómetro y el giróscopo. Solucionado \textit{in situ}, no se ha generado una issue para ello.
 El resultado de las pruebas de altitud no ha sido el deseado. No se ha conseguido que el drone mantenga la altitud de forma correcta. Si que la alcanza sin problema, pero requiere de ajustar más los parámetros del controlador PID.
 \begin{figure}[H]
	\centering
	\includegraphics[width=0.7\textwidth]{ftest2altH}
	\caption[Field Test 2. Altura alcanzada con suavidad]{El drone alcanza con relativa suavidad la altura, pero no la mantiene correctamente.}\label{fig:ftest2cripple}
\end{figure}
 
Sin embargo ha arrojado claridad sobre ciertos problemas, y mejoras que se deberían probar. 5/10.
 
 \item Calibrado del sistema de control de velocidad: La velocidad a adquirir por el drone se basa en la inclinación del mismo. Se trata, de nuevo, de un controlador PID que debe ser parametrizado. Dada la imposibilidad de mantener la altura de forma correcta, y la indudable peligrosidad de este test, se ha preferido postergar la realización de estas pruebas hasta que se haya dado con una parametrización correcta del control de altitud. 0/10.
 
 \item Calibrado del sistema de control de dirección: La dirección a seguir se basa en la orientación del drone. Se trata, de nuevo, de un controlador PID que debe ser parametrizado. Dada la imposibilidad de mantener la altura de forma correcta, y la indudable peligrosidad de este test, se ha preferido postergar la realización de estas pruebas hasta que se haya dado con una parametrización correcta del control de altitud. 0/10.
\end{itemize}


La realización de las pruebas ha dado como resultado la valoración de la release \emph{v0.2.0-alpha} como \textbf{no} lista para ser publicada. Sin lugar a dudas, requiere de muchas más pruebas en un entorno controlado antes de poder plantear la posibilidad de liberar esta versión. 

Dados los resultados de las pruebas, convendría valorar los siguientes cambios en la estructura del drone: 
\begin{itemize}
\item Reducir el peso total del dispositivo. Puede utilizarse una estructura de carbono, o similar.
\item Mejorar el control de estabilidad del drone. Parece que la controladora de vuelo Flip32 no es lo ideal para este tipo de drone tan pesado y grande. Podría valorarse la posibilidad de cambiar a una controladora más moderna, y realizar una calibración de los PID mucho más acertada. De hecho la controladora de vuelo es, en términos tecnológicos, antigua.
\item Cambiar los sensores de altitud, por uno con una tasa de adquisición mucho mayor. Este es uno de los principales problemas del cálculo del PID. El cuello de botella del sistema no está en el hecho de usar Python o de que no se trate de un sistema operativo de tiempo real, sino en la lentitud de los sensores. Los HC-SR04 requieren de un tiempo para tomar sus medidas demasiado elevado, unos $0.06s$, lo cual quiere decir que si el número de veces por segundo que se calcula el PID es muy superior al número de veces de que se dispone de información actualizada de los sensores, el cálculo del PID no es totalmente fiable, dado que trabaja con información antigua constantemente, y durante una cantidad de ciclos elevada. Por ello el uso de la componente integral queda muy limitado, ya que se dispara al no existir, aparentemente, cambios en los resultados provistos por el sensor. Podría limitarse la frecuencia del calculo de PID, aunque esto tampoco es panacea. 
\item Para evitar ecos, el `disparo' de ultrasonidos se realiza de forma secuencial, no enviando el siguiente disparo hasta que el anterior sensor haya recibido su medida. Dada la lentitud de estos sensores y que existen 5 de ellos en el cálculo de distancia a obstáculos, tomar una sola medida llevaría $0.06s * 5 = 0.3s$. Esta cantidad de tiempo, puede tornarse inaceptable en el momento en el que el sistema se desplace a una velocidad mayor, o requiera de mayor precisión. Por tanto se sugiere cambiar estos sensores por un sensor de distancia mucho más preciso y rápido, una solución basada en LIDAR debería ser valorada, o incluso en cámaras, dado el aumento de la capacidad de cómputo de las RaspberryPi.
\end{itemize}




\section{Agradecimientos}
\label{sec:agradecimientos}

Casi 90 páginas de memoria, sin contar anexos, y un montón de horas de lectura y búsqueda de información no podían pasar sin una, breve, sección de agradecimientos: 

\begin{itemize}
\item En primer lugar a Laura, que me animó a comenzar esta carrera hace cuatro años, y que nunca ha dejado de animarme.
\item A mi padre, que también es uno de mis mejores amigos, y ha sabido animarme (y aguantarme) en cualquier momento de frustración. 
\item A mis tres tutores Alejandro, Jose y César, que han gestionado este proyecto de forma impecable, y me han asesorado ante cualquier eventualidad. Me gustaría haber podido hacer más y dejarlo completado.
\item A David, mi mejor amigo, que siempre me anima a inventarme cosas nuevas, y que siempre tiene unas palabras de ánimo para mí. 
\item A Isaac, otro gran amigo, y mi cámara durante la primera sesión de pruebas, ver subsección \ref{subsec:fieldTestingv1}.
\item A Cerrajería Artística Gutiérrez, por dejarme usar un espacio en su nave en Aranda de Duero para la segunda sesión de pruebas, ver subsección \ref{subsec:fieldTestingv2}. De no ser por ellos, hoy seguiría tratando de bajar el drone de algún árbol.
\end{itemize}


















\capitulo{6}{Trabajos relacionados}

El desarrollo de este proyecto se encuentra en un ámbito bastante novedoso. La integración de \emph{drones} es cada vez más amplia en múltiples sectores, sin embargo la navegación en entornos cerrados no es tan habitual en el sector civil. 



\section{Empresas}

\subsection{TAISEI Co. LTD}
Cabe destacar el uso que está dando TAISEI Co. LTD, una compañía Japonesa, dedicada a la seguridad y limpieza, a un \emph{drone} de desarrollo propio, que pretende molestar a los trabajadores de una empresa para que no hagan horas extra.
El artículo de El Mundo, disponible en \citep{art:taisei}, y la página de la compañía disponible en \citep{wiki:taisei}.

\subsection{Erle-Robotics}
Erle Robotics es una compañía con sede en Vitoria, que se dedica a la creación de herramientas para desarrolladores de robótica. Entre sus productos se encuentra el Erle-Brain, el cual es precisamente una RaspberryPi que se encarga de controlar una controladora de vuelo, o el Erle-Copter, un \emph{drone} con características muy parecidas al desarrollado. La página de la compañía está disponible en \citep{wiki:erle}.

\subsection{FlyPulse}
FlyPulse es una empresa basada en Suecia, y dedicada a la fabricación de \emph{drones} con propósito de transporte de sistemas o elementos de asistencia médica. La página de la compañía está disponible en \citep{wiki:flypulse}.

\subsection{TUDelft} 
TUDelft es una empresa Holandesa que se hizo famosa por el desarrollo de un \emph{drone}-ambulancia capaz de transportar un desfibrilador de forma rápida y segura hasta su destino. La página del proyecto está disponible en \citep{wiki:tudelft}.


\section{Militar}
Por falta de fuentes fiables, no se citarán en este proyecto implementaciones militares dadas a la navegación autónoma en entornos cerrados.

\capitulo{7}{Conclusiones y Líneas de trabajo futuras}

En el siguiente apartado se detallan las conclusiones obtenidas del desarrollo del proyecto. A partir de estas conclusiones, se sugerirán algunas líneas de trabajo futuras que podrían complementar el proyecto, o ser proyectos completamente diferentes a este, pero con cierta relación. 

\section{Conclusiones}

\begin{itemize}
\item El objetivo del proyecto no se ha cumplido completamente. Dada la magnitud del mismo, ha sido imposible llevar a cabo todas las pruebas que habrían supuesto su finalización. Relacionado con una cuestión de tiempo y de acceso a una instalación cerrada y en la que se puedan realizar con total seguridad.
\item En múltiples asignaturas de la carrera se enuncia el principio \emph{divide y vencerás}. Sin duda, seguir este principio en el desarrollo de software se ha mostrado fundamental a la hora de escribir código fácilmente mantenible. 
\item Establecer una arquitectura genérica, que permita ir \textit{bajando} por la arquitectura desde una visión más abstracta a una visión más concreta de los diferentes problemas, puede parecer sobredimensionar y dificultar el problema, pero definitivamente es rentable cuando se quiere evitar repetir código.
\item Hacer uso de PyCharm ha facilitado mucho el desarrollo del sistema. Presenta una integración casi perfecta con despliegue en remoto, lo que ha facilitado mucho la ejecución, debug y despliegue final de la arquitectura en la RaspberryPi.
\item En teoría, todo funciona. En la práctica es cuando se descubren los verdaderos problemas. La componente física del proyecto ha aportado una visión muy crítica sobre los cursos y artículos publicados en internet. Haciendo ver que estos en muchos casos están \emph{ajustados} para que el proceso quede perfecto para su presentación.
\item Estimar la duración de las tareas es de gran dificultad, aunque se ha seguido una estrategia más bien conservadora, haciéndolas más largas de lo que en principio se preveía. 
\item Los servicios de integración continua, sobre todo aquellos relacionados con plataformas de testing, habrían sido de utilidad para evitar cometer algunos errores, sin embargo, no proveen de una forma de someter a pruebas el hardware necesario. Además, no proveen de acceso gratuito a repositorios privados, como es el caso de este proyecto.
\item Los sensores utilizados para el proyecto puede que no sean los ideales para el mismo. Si bien son baratos, y fáciles de instalar y utilizar, no son lo suficientemente rápidos en la adquisición de datos. 
\item Python es <<lento>>, pero cuando se vectoriza el cálculo haciendo uso de Numpy es casi equiparable a C.
\item Las pruebas son parte fundamental de un proyecto, sobre todo si incluye una componente física como este. De haber realizado un proyecto más sencillo, se habría dispuesto de más tiempo para llevarlas a cabo correctamente. 
\end{itemize}

\section{Líneas de trabajo futuras}

El proyecto realizado ha dado como fruto algunas líneas de trabajo por las que sería interesante continuar. No obstante, lejos de dar por finalizado (sin estarlo) este proyecto, se continuará con su desarrollo a título personal, y además se provee de una lista de proyectos derivados o relacionados con este: 

\begin{itemize}
\item ZenHub provee de un \textit{pipeline} que establece ciertas funcionalidades en pausa. Es conocido como \emph{IceBox}, y en el se irán reflejando algunas de las funcionalidades que se han dejado a la espera, y que es posible que se implementen con el tiempo. Entre las mejoras a realizar se encuentran:
\begin{itemize}
\item Implementar SLAM para no depender de un mapa prefijado.
\item Hacer uso de sensores más rápidos, o incluso cámaras para el cálculo de distancias.
\item Añadir detección de intrusos mediante diversas técnicas.
\item Desarrollar un servidor propio de WebRTC para no depender de UV4L.
\item Mejorar la interfaz web. Sin duda esta es la parte más floja del proyecto.
\end{itemize}
\item Finalizar con la etapa de pruebas, llevándolas a cabo con un \emph{drone} mejorado (sensores más rápidos, piezas a medida mediante impresión 3D... etc).
\item Implementar un framework de testing que permita hacer uso de mocks de sensores y dispositivos hardware, sin necesidad de crearlos cada vez que se quiera hacer un test.
\item Implementar un sistema de detección de intrusos y alertas al panel de control haciendo uso de la cámara disponible en la RaspberryPi. 
\item Implementar una arquitectura de enjambre de \emph{drones}. Esta arquitectura multi-agente permitiría el uso de múltiples \emph{drones} para llevar a cabo diferentes tareas. La comunicación entre los diferentes agentes para evaluar estrategias, o modificar el comportamiento de los mismos sería de gran utilidad. 
\item Implementar la creación de diferentes rutas en un equipo más potente. Si el mapa es grande y complejo, la RaspberryPi puede no ser lo suficientemente potente como para hacer esta labor de forma eficiente, de forma que sería interesante que el servidor que proporciona acceso a los \emph{drones}, pueda enviar las rutas que deben seguir estableciendo las metas a lograr.
\item Migrar la arquitectura a un RTOS. 
\item Implementar una controladora de vuelo dedicada al propósito de este proyecto, en lugar de depender de las existentes en el mercado.
\item Implementar un sistema de evasión de obstáculos no basado en un algoritmo como el VFH, sino en una CNN\footnote{Convolutional Neural Network. Este tipo de redes neuronales son muy utilizadas en procesamiento de imágenes.} haciendo uso de cámaras.
\end{itemize}

\bibliographystyle{IEEEtran}
\bibliography{bibliografia}

\end{document}
