\capitulo{6}{Trabajos relacionados}

El desarrollo de este proyecto se encuentra en un ámbito bastante novedoso. La integración de \emph{drones} es cada vez más amplia en múltiples sectores, sin embargo la navegación en entornos cerrados no es tan habitual en el sector civil. 



\section{Empresas}

\subsection{TAISEI Co. LTD}
Cabe destacar el uso que está dando TAISEI Co. LTD, una compañía Japonesa, dedicada a la seguridad y limpieza, a un \emph{drone} de desarrollo propio, que pretende molestar a los trabajadores de una empresa para que no hagan horas extra.
El artículo de El Mundo, disponible en \citep{art:taisei}, y la página de la compañía disponible en \citep{wiki:taisei}.

\subsection{Erle-Robotics}
Erle Robotics es una compañía con sede en Vitoria, que se dedica a la creación de herramientas para desarrolladores de robótica. Entre sus productos se encuentra el Erle-Brain, el cual es precisamente una RaspberryPi que se encarga de controlar una controladora de vuelo, o el Erle-Copter, un \emph{drone} con características muy parecidas al desarrollado. La página de la compañía está disponible en \citep{wiki:erle}.

\subsection{FlyPulse}
FlyPulse es una empresa basada en Suecia, y dedicada a la fabricación de \emph{drones} con propósito de transporte de sistemas o elementos de asistencia médica. La página de la compañía está disponible en \citep{wiki:flypulse}.

\subsection{TUDelft} 
TUDelft es una empresa Holandesa que se hizo famosa por el desarrollo de un \emph{drone}-ambulancia capaz de transportar un desfibrilador de forma rápida y segura hasta su destino. La página del proyecto está disponible en \citep{wiki:tudelft}.


\section{Militar}
Por falta de fuentes fiables, no se citarán en este proyecto implementaciones militares dadas a la navegación autónoma en entornos cerrados.
