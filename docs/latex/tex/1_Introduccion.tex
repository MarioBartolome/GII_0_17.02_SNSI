\capitulo{1}{Introducción}


Con el creciente uso de sistemas automatizados que ha presentado la industria en los últimos años, el nicho que los drones han pasado a ocupar crece de forma rápida haciendo que se descarten métodos muy consolidados hasta el momento. 
Vigilancia, industria cinematográfica, ocio, deportes y mensajería son solo algunos de los ejemplos en los que los drones se van abriendo camino. 

La existencia de drones cada vez más automatizados ha llegado a un punto en el que puede ser sencillo para un principiante realizar tomas de vídeo que podrían catalogarse en el ámbito profesional en una zona de difícil pilotaje.

Y este, es precisamente el reto: Lograr que un dispositivo tan complejo como un drone, que puede causar daños serios, sea sencillo de operar hasta por un principiante. 

Dada esta vertiente y los claros beneficios que un sistema así puede tener, la industria incluye cada vez más sistemas que tratan de automatizar mecanismos de seguridad, como la evasión de obstáculos o la geolocalización, sistemas de control de vuelo, como el aterrizaje y despegue, o funciones de seguimiento de objetivos, reconocimiento de imagen, planeamiento y seguimiento de rutas... etc. 

La industria parece moverse hacia sectores dedicados al exterior, donde propuestas como la evasión de obstáculos, pueden ser algo más triviales o innecesarias:

\begin{itemize}
\item Sistemas de vigilancia de incendios, en sustitución de helicópteros mucho más caros de contratar.
\item Sistemas de vigilancia de campos y cultivos.
\item Sistemas de fumigación agrícola, en sustitución de avionetas mucho más caras de mantener.
\item Sistemas de revisión de zonas de difícil acceso, en sustitución de métodos tradicionales como andamios o rápel.
\item Sistemas de grabación a nivel profesional, en sustitución de helicópteros mucho más caros de contratar.
\end{itemize}

Sin embargo parece estar obviándose el uso que se puede dar a un drone en un espacio cerrado, ya sea para acceder a zonas complejas en edificios o estructuras con gran cantidad de obstáculos, movimiento de mercancías ligeras en ciudades, en el ámbito militar, o vigilancia en general.

\noindent Incluir sistemas que provean de autonomía a un drone puede ser de tremendo beneficio para el desarrollo de un sector. 

Este proyecto propone una implementación de un sistema capaz de recorrer un espacio cerrado basándose en un plano preexistente. Evitará colisionar con elementos del entorno que detectará mediante ultrasonidos, y proporcionará una fuente de vídeo en tiempo real, su localización dentro del mapa, y dará la posibilidad de ser controlado de forma totalmente remota.

\noindent En este documento se encuentra toda la información relacionada con el Trabajo de Fin de Grado titulado \textit{Sistema de Navegación Semiautónomo en Interiores}.

\noindent En él se puede encontrar la siguiente información:
\begin{itemize}
\item \textbf{Conceptos teóricos}: Ofrecen una base teórica de la que partir, para llevar a cabo el desarrollo completo del proyecto.
\item \textbf{Técnicas y herramientas}: Se trata de las implementaciones de, ó uso dado a, los distintos conceptos teóricos anteriormente descritos.
\item \textbf{Aspectos relevantes del desarrollo del proyecto}: Proporciona información detallada que se ha tenido en cuenta durante las diferentes fases de desarrollo del proyecto.
\item \textbf{Trabajos relacionados}: Se trata de una lista, junto con una breve descripción, de los diferentes proyectos, papers o trabajos relacionados con el proyecto llevado a cabo.
\item \textbf{Conclusiones y líneas de trabajo futuras}: Detalla una serie de posibles mejoras, modificaciones e incluso derivaciones, que pueden surgir del proyecto realizado.
\end{itemize}
