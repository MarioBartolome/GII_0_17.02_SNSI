\apendice{Especificación de Requisitos}

\section{Introducción}

En este anexo se recogen las especificaciones de requisitos requeridas para este proyecto. Estas definirán el comportamiento esperado del sistema. 

\section{Objetivos generales}

En este proyecto se ha tratado de llevar a cabo los siguientes objetivos de carácter general: 

\begin{itemize}
\item Diseñar un drone capaz de recorrer un espacio, en el que existan obstáculos, de forma segura.
\item Diseñar un sistema de acceso al drone de forma segura. Tanto para controlarlo de forma remota, como para activa los mecanismos de control automatizados.
\item Diseñar una interfaz web que permita la visualización en tiempo real de la cámara del drone, así como su control remoto por un operador.
\end{itemize}


\section{Catalogo de requisitos}
Los siguientes requisitos se derivan de los objetivos generales arriba dispuestos: 

\subsubsection{Requisitos Funcionales}
\begin{itemize}
\item[\textbf{RF-1}] Acceso al sistema: El usuario debe poder iniciar sesión en el sistema.

\item[\textbf{RF-2}] Cierre de sesión: El usuario debe poder cerrar sesión en el sistema.

\item[\textbf{RF-3}] Gestión del sistema: El usuario debe ser capaz de gestionar el sistema completo que hay a bordo del drone.

	\begin{itemize}

			\item[\textit{RF-3.1}] Gestión del streaming de vídeo: El usuario debe poder iniciar/detener el streaming de vídeo.
			\item[\textit{RF-3.2}] Grabaciones de vídeo: El usuario debe poder iniciar/detener y descargar las grabaciones de vídeo.
			\item[\textit{RF-3.3}] Gestión de controles del drone: El usuario debe poder activar/desactivar el control manual del dispositivo.
				\begin{itemize}
					\item[RF-3.3.1] Seguridad automatizada: El drone debe presentar mecanismos de autopreservación. Debe protegerse frente a obstáculos.
				\end{itemize}
			\item[\textit{RF-3.4}] Gestión de control automático del drone: El usuario debe poder activar/desactivar los mecanismos de control automatizado.

		\item[\textit{RF-3.5}] Registro de acciones: Las acciones llevadas a cabo por el usuario deben guardarse en un registro.
		\end{itemize}
		
\item[\textbf{RF-4}] Gestión del backend: El usuario debe poder acceder a la equipo que controla el backend.
			\begin{itemize}
				\item[\textit{RF-4.1}] Gestión del sistema operativo: El usuario debe poder llevar a cabo las tareas de gestión del sistema operativo como crea conveniente.
			\end{itemize}		
	

\item[\textbf{RF-5}] Registro de sucesos: Intentos de intrusión deben ser registrados.

\item[\textbf{RF-6}] Reconocimiento de dispositivos: La aplicación web debe ser capaz de reconocer un dispositivo (joystick o emisora) conectado al equipo, para llevar a cabo el control del drone.


\end{itemize}


\subsubsection{Requisitos no funcionales}
\begin{itemize}
\item[\textbf{RNF-1}] Usabilidad: El frontend ha de ser sencillo de utilizar e intuitiva. En el caso de producirse errores, estos deben ser claramente explicados, o registrados.
\item[\textbf{RNF-2}] Rendimiento: Tanto el frontend, como el backend deben funcionar de forma eficiente. Retardos elevados podrían dar lugar a problemas muy serios en la ejecución de las ordenes del usuario.
\item[\textbf{RNF-3}] Capacidad y Escalabilidad: Tanto el frontend como el backend deben ser escalables. El frontend debe ser capaz de soportar una asiduidad grande de usuarios, así como permitir añadir múltiples agentes a cada usuario. El backend, debe permitir añadir controladores sin que haya que reestructurar todo el sistema.
\item[\textbf{RNF-4}] Disponibilidad: El sistema al completo debe ser robusto. El frontend debe estar disponible para su uso el mayor tiempo que sea posible. El backend debe ser capaz de recuperarse de errores, y determinar acciones a tomar en el caso de los mismos.
\item[\textbf{RNF-5}] Seguridad: Tanto el acceso al frontend como al backend deben hacerse de forma segura. En ningún momento el drone debe guardar imágenes de la zona vigilada. El almacenamiento de contraseñas se hará de forma cifrada, y la comunicación entre el backend y el frontend deberá ser segura.
\end{itemize}

\section{Especificación de requisitos}

Esta sección detallará el diagrama de casos de uso, y detallará cada uno de ellos. 



\afterpage{
\clearpage
\begin{landscape}
\begin{figure}[H]
	\centering
	\includegraphics[width=1.35\textwidth]{caseUseDiag}
	\caption[Diagrama de Casos de Uso]{Diagrama de Casos de Uso.}\label{fig:caseUseDiag}
\end{figure}

\end{landscape}
\clearpage
}


\subsection{Actores}

\begin{itemize}
\item \textbf{Vigiliante:} Actor a cargo de la gestión del sistema en producción. Interactúa con el FrontEnd.
\item \textbf{Admin:} Actor a cargo de la gestión del sistema en desarrollo o en caso de resolución de problemas en producción. Interactúa con el BackEnd.
\end{itemize}

\subsection{Casos de Uso}

\begin{table}[H]
	\begin{center}
	\rowcolors {2}{gray!15}{}
		\begin{tabular}{m{3cm} | m{10cm}}\hline
			\toprule
			\textbf{CU-01} & \textbf{Acceso al sistema}\\
			\otoprule
			\textbf{Actor} & Vigilante\\
			\textbf{Requisitos asociados} & RF-1, RF-5\\
			\textbf{Descripción} & Permite al usuario iniciar sesión en el sistema vía web.\\
			\textbf{Precondiciones} & La base de datos está activa. El usuario dispone de cuenta activa.\\
			\textbf{Acciones} & \begin{enumerate}
											\item El usuario accede a la página web.
											\item El usuario introduce sus credenciales.
											\item Se muestra la página principal con el logo y los botones asociados al streaming de vídeo.
											\end{enumerate}\\
			
			\textbf{Postcondición} & El sistema habrá asignado al usuario el drone correspondiente. \\
			\textbf{Excepciones} & \begin{itemize}
												\item Contraseña/Usuario incorrectos (mensaje).
												\item No existen drones asociados (mensaje).
												\end{itemize}\\
			\textbf{Importancia} & Alta.\\
			\hline
			\bottomrule
		\end{tabular}
		\caption{CU-01. Acceso al sistema}
		\label{tb:CU01}
	\end{center}
\end{table}



\begin{table}[H]
	\begin{center}
	\rowcolors {2}{gray!15}{}
		\begin{tabular}{m{3cm} | m{10cm}}\hline
			\toprule
			\textbf{CU-02} & \textbf{Cierre de sesión}\\
			\otoprule
			\textbf{Actor} & Vigilante\\
			\textbf{Requisitos asociados} & RF-2, RF-5\\
			\textbf{Descripción} & Permite al usuario cerrar sesión en el sistema vía web.\\
			\textbf{Precondiciones} & La base de datos está activa. El usuario dispone de cuenta activa. El usuario tiene iniciada una sesión.\\
			\textbf{Acciones} & \begin{enumerate}
											\item El usuario accede a la aplicación web.
											\item El usuario pulsa sobre el enlace `Logout' situado en la parte superior izquierda.
											\item Se muestra la página de inicio de sesión con un mensaje de despedida.
											\end{enumerate}\\
			
			\textbf{Postcondición} & El sistema habrá dado por cerrada la sesión de usuario. Se redirige al usuario a la página de inicio de sesión.\\
			\textbf{Excepciones} & El usuario no tenía sesión activa. (redirección a login).\\
			\textbf{Importancia} & Alta.\\
			\hline
			\bottomrule
		\end{tabular}
		\caption{CU-02. Cierre de sesión}
		\label{tb:CU02}
	\end{center}
\end{table}

\begin{table}[H]
	\begin{center}
	\rowcolors {2}{gray!15}{}
		\begin{tabular}{m{3cm} | m{10cm}}\hline
			\toprule
			\textbf{CU-03} & \textbf{Gestión del sistema}\\
			\otoprule
			\textbf{Actor} & Vigilante\\
			\textbf{Requisitos asociados} & RF-1, RF-3\\
			\textbf{Descripción} & Permite al usuario gestionar las acciones del drone vía web.\\
			\textbf{Precondiciones} & La base de datos está activa. El usuario dispone de cuenta activa. El usuario tiene iniciada una sesión. El usuario tiene asignado un drone.\\
			\textbf{Acciones} & \begin{enumerate}
											\item El usuario conecta un dispositivo con el número de canales necesarios. 
											\item Se informa sobre el reconocimiento de dicho dispositivo y se muestran los valores de los canales activos.
											\item Se muestra el botón de activación del control manual.
											\end{enumerate}\\
			
			\textbf{Postcondición} & El sistema habrá asignado al usuario el drone correspondiente. La aplicación web habrá reconocido el dispositivo conectado. Se dará la posibilidad de activar el control manual.\\
			\textbf{Excepciones} & Dispositivo no reconocido.\\
			\textbf{Importancia} & Alta.\\
			\hline
			\bottomrule
		\end{tabular}
		\caption{CU-03. Gestión del sistema}
		\label{tb:CU03}
	\end{center}
\end{table}


\begin{table}[H]
	\begin{center}
	\rowcolors {2}{gray!15}{}
		\begin{tabular}{m{3cm} | m{10cm}}\hline
			\toprule
			\textbf{CU-04} & \textbf{Gestión del streaming de vídeo}\\
			\otoprule
			\textbf{Actor} & Vigilante\\
			\textbf{Requisitos asociados} & RF-3, RF-3.1 RF-3.5\\
			\textbf{Descripción} & Permite al usuario activar/desactivar el feed de vídeo proveniente del drone.\\
			\textbf{Precondiciones} & La base de datos está activa. El usuario dispone de cuenta activa. El usuario tiene iniciada una sesión. El usuario tiene asignado un drone.\\
			\textbf{Acciones} & \item El usuario pulsa sobre el botón `Start/Stop Streaming'\\
			
			\textbf{Postcondición} & La aplicación web inicia/detiene el vídeo.\\
			\textbf{Excepciones} &  Drone no disponible (mensaje).\\
			\textbf{Importancia} & Alta.\\
			\hline
			\bottomrule
		\end{tabular}
		\caption{CU-04. Gestión del streaming de vídeo}
		\label{tb:CU04}
	\end{center}
\end{table}


\begin{table}[H]
	\begin{center}
	\rowcolors {2}{gray!15}{}
		\begin{tabular}{m{3cm} | m{10cm}}\hline
			\toprule
			\textbf{CU-05} & \textbf{Grabación de vídeo}\\
			\otoprule
			\textbf{Actor} & Vigilante\\
			\textbf{Requisitos asociados} & RF-3, RF-3.2 RF-3.5\\
			\textbf{Descripción} & Permite al usuario grabar el feed de vídeo proveniente del drone.\\
			\textbf{Precondiciones} & La base de datos está activa. El usuario dispone de cuenta activa. El usuario tiene iniciada una sesión. El usuario tiene asignado un drone.\\
			\textbf{Acciones} & \item El usuario pulsa sobre el botón `Record Video'\\
			
			\textbf{Postcondición} & La aplicación web comienza con la grabación del feed de vídeo.\\
			\textbf{Excepciones} &\item Feed de vídeo no disponible (mensaje).\\
			\textbf{Importancia} & Baja.\\
			\hline
			\bottomrule
		\end{tabular}
		\caption{CU-05. Grabación de vídeo}
		\label{tb:CU05}
	\end{center}
\end{table}

\begin{table}[H]
	\begin{center}
	\rowcolors {2}{gray!15}{}
		\begin{tabular}{m{3cm} | m{10cm}}\hline
			\toprule
			\textbf{CU-06} & \textbf{Control manual del drone}\\
			\otoprule
			\textbf{Actor} & Vigilante\\
			\textbf{Requisitos asociados} & RF-3, RF-3.3 RF-3.5, RF-3.3.1, RF-6\\
			\textbf{Descripción} & Permite al usuario controlar el drone de forma manual.\\
			\textbf{Precondiciones} & La base de datos está activa. El usuario dispone de cuenta activa. El usuario tiene iniciada una sesión. El usuario tiene asignado un drone. El usuario ha conectado un dispositivo adecuado.\\
			\textbf{Acciones} & El usuario pulsa sobre el botón `Enable/Disable Manual'\\
											
			\textbf{Postcondición} & La aplicación web solicita la activación del control manual al sistema. El sistema se conecta al drone y transmite la información proveniente de la aplicación web.\\
			\textbf{Excepciones} & Drone no disponible (mensaje).\\
			\textbf{Importancia} & Alta.\\
			\hline
			\bottomrule
		\end{tabular}
		\caption{CU-06. Control manual}
		\label{tb:CU06}
	\end{center}
\end{table}

\begin{table}[H]
	\begin{center}
	\rowcolors {2}{gray!15}{}
		\begin{tabular}{m{3cm} | m{10cm}}\hline
			\toprule
			\textbf{CU-07} & \textbf{Gestión de controles automáticos}\\
			\otoprule
			\textbf{Actor} & Vigilante\\
			\textbf{Requisitos asociados} & RF-3, RF-3.4 RF-3.5\\
			\textbf{Descripción} & Permite al usuario activar/desactivar los mecanismos de control automatizado del drone.\\
			\textbf{Precondiciones} & La base de datos está activa. El usuario dispone de cuenta activa. El usuario tiene iniciada una sesión. El usuario tiene asignado un drone.\\
			\textbf{Acciones} & \begin{enumerate}
											\item El usuario selecciona los mecanismos de control automatizado que desea emplear.
											\item El usuario pulsa sobre el botón `Enable auto'
											\end{enumerate}\\
											
			\textbf{Postcondición} & La aplicación web solicita la activación de los controles automatizados al sistema. El sistema se conecta al drone y transmite la información proveniente de la aplicación web.\\
			\textbf{Excepciones} & \begin{itemize}
												\item Drone no disponible (mensaje).
												\item Mecanismo automatizado no reconocido (mensaje).
												\end{itemize}\\
			\textbf{Importancia} & Media.\\
			\hline
			\bottomrule
		\end{tabular}
		\caption{CU-07. Gestión de controles automáticos}
		\label{tb:CU07}
	\end{center}
\end{table}


\begin{table}[H]
	\begin{center}
	\rowcolors {2}{gray!15}{}
		\begin{tabular}{m{3cm} | m{10cm}}\hline
			\toprule
			\textbf{CU-08} & \textbf{Gestión del backend}\\
			\otoprule
			\textbf{Actor} & Admin\\
			\textbf{Requisitos asociados} & RF-4, RF-5\\
			\textbf{Descripción} & Permite al usuario gestionar el equipo que controla el drone.\\
			\textbf{Precondiciones} & El usuario dispone de credenciales de acceso al sistema.\\
			\textbf{Acciones} & \begin{enumerate}
											\item El usuario inicia sesión en el sistema.
											\item Se informa sobre el registro de intentos de acceso no autorizado.
											\item Se muestra prompt del sistema a la espera de órdenes.
											\end{enumerate}\\
			
			\textbf{Postcondición} & El sistema operativo habrá registrado el inicio de sesión del usuario. El sistema operativo habrá iniciado sesión con el usuario solicitado.\\
			\textbf{Excepciones} & Acceso denegado, pubkey. (mensaje)\\
			\textbf{Importancia} & Alta.\\
			\hline
			\bottomrule
		\end{tabular}
		\caption{CU-08. Gestión del backend.}
		\label{tb:CU08}
	\end{center}
\end{table}


